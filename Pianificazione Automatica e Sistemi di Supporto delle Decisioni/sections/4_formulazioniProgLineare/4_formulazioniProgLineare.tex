\newpage
\section{Formulazioni equivalenti di un problema di programmazione lineare}


% Problema in FORMA GENERALE
\subsection{Problema in FORMA GENERALE}

In generale nella \hl{zona ammissibile} diciamo:

$$X={x \in R^n : A_x \geq b, D_x = l, x_j \geq 0}\ \ \ (*)$$
$$\forall\ \ \ j \in J \subseteq \{1,2,...,n\}$$

dove \hl{definiamo la possibilita' di vincoli di $\geq$, = e variabili $\geq$ 0}.

La funzione obiettivo è definita da: $$z = c_x\ \ \ t.c.\ \ \ z = \min z, x \in X$$

che rappresenta il \hl{problema espresso in forma generale}.


% Problema in FORMA CANONICA
\subsection{Problema in FORMA CANONICA}

Se in (*) abbiamo:

\begin{itemize}
	\item $D = 0$
	\item $J = \{1,2,...,n\}$
\end{itemize}

allora il \hl{problema si dice in forma canonica}: $$min_x\ \ \{z = c_x : A_x \geq b, x \geq 0\}$$


% Problema in FORMA STANDARD
\subsection{Problema in FORMA STANDARD}

Se in (*) abbiamo:

\begin{itemize}
	\item $A = 0$
	\item $J = {1,2,..,n}$
\end{itemize}

allora il problema è: $$min_x\ \ \{z = c_x: D_x = l, x \geq 0\}$$

Possiamo sempre \hl{ricondurci tramite trasformazioni alla forma standard}.


% Terminologia
\subsection{Terminologia}

Nella forma standard abbiamo che:

\begin{itemize}
	\item \hl{z}: la \textbf{funzione obbiettivo} per la quale trovare il valore minimo
	\item \hl{A}: \textbf{matrice dei vincoli}
	\item \hl{D}: \textbf{matrice dei coefficienti}, matrice di dimensione m x n
	\item \hl{l}: \textbf{vettore dei termini noti} vettore colonna 
	\item \hl{c}: \textbf{detto vettore dei coefficienti di costo}, vettore di riga
\end{itemize}


% Trasformazioni per ricondursi alla forma standard
\subsection{Trasformazioni per ricondursi alla forma standard}

\begin{enumerate}
	\item \hl{variabili non vincolate di segno}: $$x_j t.c. j \notin J$$

		per trasformarla possiamo \hl{sostituire a $x_j$ la somma algebrica di 2 variabili non negative}: $$x_j = x_j^+ - x_j^-$$ $$\forall\ \ \ x_j^+ \geq 0, x_j^- \geq 0$$

		Se abbiamo $k$ ($\leq n$) variabili non vincolate in segno, possiamo evitare di introdurre $k$ coppie di variabili non negative. È possibile considerare una variabile $x_0 \geq 0$ e sostituire la generica variabile non vincolata di segno con $x_j = x_j^+ - x_0$. \hl{Cosi' introduciamo "solo" $k+1$ variabili}.
	
	\item \hl{vincoli non espressi in forma di uguaglianza ($\leq$)}
		
		$$\sum_{j=i}^n a_{ij} x_j \leq b_i$$
	
		presa la \hl{variabile di Slack}: $S_i \geq 0$, avremo:
	
		$$\sum_{j=i}^n a_{ij} x_j + S_i = b_i$$
	
	
		questa variabile misura lo Slack che esiste per far si che \hl{il vincolo sia rispettato per uguaglianza} o no (vincolo $>$ 0)
		
	\item \hl{vincoli non espressi in forma di uguaglianza ($\geq$)}
	
		$$\sum_{j=i}^n a_{ij} x_j \geq b_i$$
		
		presa una \hl{variabile di Surplus}: $S_i \geq 0$, avremo:
		
		$$\sum_{j=i}^n a_{ij} x_j - S_i = b_i$$
		
	\item \hl{trasformazione di vincoli da uguaglianza in disuguaglianza} sostituendo:

		$$\sum_{j=i}^n a_{ij} x_j = b_i$$
		
		\hl{sostituendolo a}:
		
		$$\sum_{j=i}^n a_{ij} x_j \geq b_i$$
		$$\sum_{j=i}^n a_{ij} x_j \leq b_i$$
		
	\item \hl{funzione obiettivo}:
	
		Se la f.o. è $\max z = c_x$, si può trasformare:
	
		$$\max z = -\min {-z}$$

	
\end{enumerate}


% Esempio 1
\subsection{Esempio 1}

\begin{enumerate}
	\item DATI
	
		f.o.: $\min z= x_1 + 2 x_2$

		vincoli:

		\begin{itemize}
			\item $6x_1 + 4x_2 \leq 24$
			\item $4x_1 + 8x_2 \leq 32$
			\item $x_2 \geq 3$
			\item $x_1, x_2 \geq 0$
		\end{itemize}
		
	\item TRASFORMAZIONE IN FORMA STANDARd
	
		Per il primo vincolo \textbf{del tipo $\leq$, aggiungiamo una variabile non negativa} (slack):

		$$6x_1 + 4x_2 + x_3 = 24$$

		per il secondo vincolo operiamo in maniera analoga :

		$$4x_1 + 8x_2 + x_4 = 32$$

		per il terzo vincolo \textbf{del tipo $\geq$, aggiungiamo una variabile ausiliaria negativa}:
		
		$$x_2 - x_5 = 3$$

		vincolo sulle variabili:

		$$x_1, x_2, x_3, x_4, x_5 \geq 0$$

		quindi nella formulazione standard avremo $j = 1,2,3,4,5$ quando in precedenza avevamo: $j = 1, 2$
	
\end{enumerate}


% Esempio 2
\subsection{Esempio 2}

\begin{enumerate}
	\item DATI
	
		f.o.: $\max z = z_1 + z_2$

		vincoli:

		\begin{itemize}
			\item $8x_1 + 6x_2 \geq 48$
			\item $5x_1 + 10x_2 \geq 50$
			\item $13x_1 + 10x_2 \leq 130$
			\item $x_1 \geq 0$
		\end{itemize}
		
	\item TRASFORMAZIONE IN FORMA STANDARD
	
		Dato che non abbiamo vincoli su $x_2$ poniamo:
		
		$$x_2 = x_2^+ - x_2^-$$
		
		la f.o. sarà:
		
		$$\max z = -\min -z = -x_1 -x_2 = -x_1 -x_2^+ + x_2^-$$
		
		invece i vincoli:
		
		\begin{itemize}
			\item $8x_1 + 6x_2^+ - 6x_2^- - x_3 = 48$
			\item $5x_1 + 10_2^+ - 10x_2^- - x_4 = 50$
			\item $13x_1 + 10_2^+ - 10x_2^- + x_5 = 130$
			\item $x_1, _2^+, x_2^-, x_3, x_4, x_5 \geq 0$
		\end{itemize}
	
\end{enumerate}

