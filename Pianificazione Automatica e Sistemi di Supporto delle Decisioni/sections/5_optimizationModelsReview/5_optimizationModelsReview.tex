\newpage
\section{Optimization models review}

% Scheduling
\subsection{Scheduling}

Nell'ambito dei problemi dello scheduling abbiamo degli elementi ben specificati:

\begin{itemize}
	\item \hl{task/job} già assegnati
	\item $n$ \hl{macchine/processori}
	\item potremmo attrezzare le macchine con dei \hl{tools}
\end{itemize}

Intendiamo \hl{allocare i tasks alla macchine in "overtime"} quindi capire anche la \hl{fascia temporale nella quale eseguire il task}. Potrebbe esserci un unico tempo di esecuzione oppure un task può avere dei tempi di esecuzione differenti su macchine differenti. 

L'output sarà un diagramma di Ganth.

\hl{I task possono avere degli istanti di rilascio} dove non potrebbe essere rilasciato dopo un certo istante di tempo (\hl{ready time}).

Possono esserci delle \hl{relazioni di precedenza tra i tasks}. Quindi non posso effettuare un task se prima non ho concluso l'altro.

Il diagramma mi dice nel tempo a che macchina è associato quale task ed in quali intervalli di tempo e con quale tool.


% Project scheduling
\subsection{Project scheduling}

Per progetto intendiamo un \hl{insieme di tasks che sono realizzati al fine di raggiungere un goal}. La caratteristica di un progetto è che nel complesso le attività non sono mai state eseguite in precedenza.

Le \hl{caratteristiche di un progetto} sono:


\begin{itemize}
	\item \textbf{durata delle attività} che nota
	\item ha \textbf{a capo un Project Manager}: responsabile del progetto e dei tempi di realizzazione, costi di produzione, ecc...
\end{itemize}



in genre in project manager usano dei metodi per tener sotto ontrollo il progetto (es: microsoft project) per esempoio degli applicativi che suppostano i p.m. nella schedulaizone e rischedulaizone dei progetti 



i lpiù semplice modello di modellazione:

un porgetto e1 rappresentato da diverse attività:

expected puration




\begin{table}[h!]
	\begin{center}
	\begin{tabular}{|c | c c |} 
			\hline
			Attività & Durata stimata d_i & Predecessori \\ [0.5ex]
			\hline
 			1 & 10 & - \\
			2 & 10 & - \\
			3 & 10 & 1 \\
			4 & 10 & 1, 2 \\
			\hline
		\end{tabular}
	\end{center}
	\caption{Tabella delle ore di lavoro}
	\label{taborelav}
\end{table}


questi problemi vengono rapresntati tramite diagramma aciclico:

rapresntazione activity on node (AoN):

dove abbiamo degli "archi" che rappresentano le predecessioni 

(la tabella mi viene fornita dal project manager dopo aver parlato con delgi esperti)



a volte si introducono dei vertici fittizzi che sono lo start e l'end

start: lo colleghiamo tutte le atticità che non hano predecessori
end: ci colleghiamo tutte le attivtà finali




vedremo come li p.m. può usare delle risorse per accellerare delle attività


prima cosa: cariabili decisionali: nel nostro caso è lo start time. ipotizziamo che i lprogetto inizi al tempo t = 0, quidni per ogni task abbiamo che s_i >= 0 \forall i \in TASKS dove s_i è lo staart time del task i che appartiene a TASKS

in più possiamo definire T >= 0 che è il tempo di completamento del progetto (complition time)


in questo modello vogliamo minimizzare il complation time : min z = T

cioè: z = 1T + 0s_1 + 0s_2 + ...sucamiento


per quanto riguarda le relaizoni di precedenza:

p_{ij} = { (sistema) 1 <=> i è predecessore di j, 0 altrimenti

con p_{ij} una matrice costate e binaria:

p = 

0 0 1 1
0 0 0 1
0 0 0 0
0 0 0 0




VINCOLI DI PRECEDENZA:

T è un maggiorante del tempo di completamento delle task, quindi:

s_i + d_i <= T

p_j (s_i + d_i) <= s_j \forall i \in TASKS, j \in TASKS
se p_ij = 1 allora i e1 predecessore di j quindi il tempo di dinizion del task j deve sessere duccessivo o uguale al task i cioe1 s_i + d_i


se invece i non e1 predecessore: p_ij = 0 quindi avremo che 0 <= s_j allora il vincolo è ridondante dato che e1 un dato che già abbiamo


questo vale per un generico numero di task, mi serve pero un lingiaggi odi modellaizone per scrivere il modello in forma compatta per avere dati e modello divisi


per semplifia l'ultimo vicolo possoiamo scrivere 
s_i + d_i <= s_j \forall i \in TASKS, j \in TASKS, p_{ij} >= 0





esempio modello espanso per problemi di istanza con i dati di sopra:

min z = T

s_1 + 10 <= T
s_2 + 10 <= T
s_3 + 10 <= T
s_4 + 10 <= T

s_1 + 10 <= s_3 (p_{13} = 1)
s_1 + 10 <= s_4 (p_{14} = 1)
s_2 + 10 <= s_4 (p_{24} = 1)

s_1, s_2, s_3, s_4 >= 0
T >= 0










il project manager ha un budget per poter velocizzare il porgetto

possiamo comprimere un'attvità. se considero un task "i" con durata ovviamente non costante, sarà allora data da:

d_i^N(valore nominale) durata attivatà orrspondete ad un utilizzo di risorse non extra.

accade però che se usao piu risrose l'attivtà può essere ridotta. un andamento sempplice èuno linare dove all'aumentare delle risorse la durata si riduce in. modo linare. il che e1 vero solo per un certo punto dato che ci può essere un vincolo per il quale sotto un valolre inimo d_i^m non si può scendere perchè l'attivtà non è piú comprimibile


le 3 risorse alle quali si possono far riferimento sono le 3M Man, Machine, Money.


quindi abbiamo che d_i = d_i^N

w = pendenza: riduzione della durata del task "i" per unità di extra risorse (mesi di lavoro / k euro)

abbiamo allora che: d_i = d_i^N - w_i x_i

con x_i valore in euro che sono stati usati

ma dovrò esprimere un vincolo per la durata minima del problema:

d_i = d_i^N - w_i x_i >= d_i^m


tra i nostri dati avremo allora un extra budget "B"




in un modello avremo allora:

funzione obiettivo: $\min z = T$

s.v.:

\begin{itemize}
	\item $s_i + d_i^N - w_i x_i <= T\ \ \ \forall\ \ \ i \in TASKS$
	\item $s_i + d_i^N - w_i x_i <= s_j\ \ \ \forall\ \ \ i, j \in TASKS, p_{ij} = 1$
	\item $d_i^N - w_i x_i >= d_i^m\ \ \ \forall\ \ \ i \in TASKS$
	\item $\sum_{i \in TASKS} x_i <= B$
	\item $T >= 0$
	\item $s_i >= 0\ \ \ \forall\ \ \ i \in TASKS$
	\item $x_i >= 0\ \ \ \forall\ \ \ i \in TASKS$
\end{itemize}











LOW SIZING MODELS:

in genere usati da aziende manifatturiere. 

nel nostrao esempi otutto il sistema produttivo è modellato come una black box (processo produttivo) poi abbiamo uno stockaggio e a valle in cliente finale

supponiamo che il tasso di domanda del cliente sia sostante nel tempo. questo tasso di domanda "d" si può esprimere in base al tipo di prodotto.

queta line a di produzione può produrre diversi articoli cioe1 con dei prodotti con diverse differenze tipo yogurt a vari gusti ma poi ogni volta avremo un costo di setup per poter pulire le macchine questo costo fisso "k" per la produzione di un lotto.

la porduzione e1 a lotti quindi se ho un livello delle scorte = 0 allora avremoche la dimnsione del letto e1 q allora il livello salta da 0 a q 


img 


poiche la domanda e1 ipotizzata costante allora nei giorni successivi il livello delle scrorte diminuisce linearmente con un pendenza che dimende dal tasso di domanda


abbiamo costi fissi ogni volta che produciamo un lotto paghiamo k.

abbiamo anche dei costi di scorte e quindi pagherò un costo "h". ovviamente le scrote no staranne ferme in mangazzino ma gireranno e si scambierano tra loro quindi in media avremo q/2 unità di prodotto.

abbaimo aun costo di questa giacenza media per un certo intervallo di tempo: h detto csoto medio di stockaggio.



caso estermo: gestione di tipo just in time: produco solo sotto commissione del cliente

questo comporta un livello di scorte molto basso (q) quidi il livello medio delle cscorte e1 molto basso e quindi meno costi di inventory. però abbiamo un maggioramnto dei costi del setup. a meno che non costino poso i setup perche molto efficienti. pago k più volte


caso ?: produco un quantitativo q pari alla domanda annua e quidni con solo quella soddisfo per tutto l'anno. quindi abbiamo che paghiamo k una sla volta ma ho grandi costi di stockaggio



per scrivere il modello di ottimizzazione scriviamo:

1. scrivo variabili decisionali(variabile matematica per descriver la mia decisoione): q rappresnta la dimnzione dei miei lotti stockati

2. scrivere funzione obiettivo: obiettvo: costo totale annuale composto dal costo di scorta e quello di setup

z = k d/q (domanda annuale/numero di lotti per anno) + h q/2 


h costo medio per un pallet per un anno



per la soluzione ottima, faccio il gradiente:

dz/dq = 0 <=> -k d/q^2 + h/2 = 0


q^* = sqrt({2kd}/h)formula del lotto economico (economico: nel senso che fa minimare i costi)



img


