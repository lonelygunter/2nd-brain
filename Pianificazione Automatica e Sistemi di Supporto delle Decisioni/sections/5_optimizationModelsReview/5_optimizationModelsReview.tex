\newpage
\section{Optimization models review}

% Scheduling
\subsection{Scheduling}

Nell'ambito dei problemi dello scheduling abbiamo degli elementi ben specificati:

\begin{itemize}
	\item \hl{task/job} già assegnati
	\item $n$ \hl{macchine/processori}
	\item potremmo attrezzare le macchine con dei \hl{tools}
\end{itemize}

Intendiamo \hl{allocare i tasks alla macchine in "overtime"} quindi capire anche la \hl{fascia temporale nella quale eseguire il task}. Potrebbe esserci un unico tempo di esecuzione oppure un task può avere dei tempi di esecuzione differenti su macchine differenti. 

L'output sarà un diagramma di Ganth.

\hl{I task possono avere degli istanti di rilascio} dove non potrebbe essere rilasciato dopo un certo istante di tempo (\hl{ready time}).

Possono esserci delle \hl{relazioni di precedenza tra i tasks}. Quindi non posso effettuare un task se prima non ho concluso l'altro.

Il diagramma mi dice nel tempo a che macchina è associato quale task ed in quali intervalli di tempo e con quale tool.


% Project scheduling
\subsection{Project scheduling}

Per progetto intendiamo un \hl{insieme di tasks che sono realizzati al fine di raggiungere un goal}. La caratteristica di un progetto è che nel complesso le attività non sono mai state eseguite in precedenza.

Le \hl{caratteristiche di un progetto} sono:


\begin{itemize}
	\item \textbf{durata delle attività} che nota
	\item ha \textbf{a capo un Project Manager}: responsabile del progetto e dei tempi di realizzazione, costi di produzione, ecc...
\end{itemize}



in genre in project manager usano dei metodi per tener sotto ontrollo il progetto (es: microsoft project) per esempoio degli applicativi che suppostano i p.m. nella schedulaizone e rischedulaizone dei progetti 



i lpiù semplice modello di modellazione:

un porgetto e1 rappresentato da diverse attività:

expected puration




\begin{table}[h!]
	\begin{center}
	\begin{tabular}{|c | c c |} 
			\hline
			Attività & Durata stimata d_i & Predecessori \\ [0.5ex]
			\hline
 			1 & 10 & - \\
			2 & 10 & - \\
			3 & 10 & 1 \\
			4 & 10 & 1, 2 \\
			\hline
		\end{tabular}
	\end{center}
	\caption{Tabella delle ore di lavoro}
	\label{taborelav}
\end{table}


questi problemi vengono rapresntati tramite diagramma aciclico:

rapresntazione activity on node (AoN):

dove abbiamo degli "archi" che rappresentano le predecessioni 

(la tabella mi viene fornita dal project manager dopo aver parlato con delgi esperti)



a volte si introducono dei vertici fittizzi che sono lo start e l'end

start: lo colleghiamo tutte le atticità che non hano predecessori
end: ci colleghiamo tutte le attivtà finali




vedremo come li p.m. può usare delle risorse per accellerare delle attività


prima cosa: cariabili decisionali: nel nostro caso è lo start time. ipotizziamo che i lprogetto inizi al tempo t = 0, quidni per ogni task abbiamo che s_i >= 0 \forall i \in TASKS dove s_i è lo staart time del task i che appartiene a TASKS

in più possiamo definire T >= 0 che è il tempo di completamento del progetto (complition time)


in questo modello vogliamo minimizzare il complation time : min z = T

cioè: z = 1T + 0s_1 + 0s_2 + ...sucamiento


per quanto riguarda le relaizoni di precedenza:

p_{ij} = { (sistema) 1 <=> i è predecessore di j, 0 altrimenti

con p_{ij} una matrice costate e binaria:

p = 

0 0 1 1
0 0 0 1
0 0 0 0
0 0 0 0




VINCOLI DI PRECEDENZA:

T è un maggiorante del tempo di completamento delle task, quindi:

s_i + d_i <= T

p_j (s_i + d_i) <= s_j \forall i \in TASKS, j \in TASKS
se p_ij = 1 allora i e1 predecessore di j quindi il tempo di dinizion del task j deve sessere duccessivo o uguale al task i cioe1 s_i + d_i


se invece i non e1 predecessore: p_ij = 0 quindi avremo che 0 <= s_j allora il vincolo è ridondante dato che e1 un dato che già abbiamo


questo vale per un generico numero di task, mi serve pero un lingiaggi odi modellaizone per scrivere il modello in forma compatta per avere dati e modello divisi


per semplifia l'ultimo vicolo possoiamo scrivere 
s_i + d_i <= s_j \forall i \in TASKS, j \in TASKS, p_{ij} >= 0





esempio modello espanso per problemi di istanza con i dati di sopra:

min z = T

s_1 + 10 <= T
s_2 + 10 <= T
s_3 + 10 <= T
s_4 + 10 <= T

s_1 + 10 <= s_3 (p_{13} = 1)
s_1 + 10 <= s_4 (p_{14} = 1)
s_2 + 10 <= s_4 (p_{24} = 1)

s_1, s_2, s_3, s_4 >= 0
T >= 0

