\section{Introduzione all'ottimizzazione matematica - 30.09.22}

% Introduzione
\subsection{Introduzione}

Partiamo da un \hl{insieme di formule ed equazioni che modelleranno il problema}. Con questo modello proviamo a trovare una \hl{soluzione} al nostro problema \hl{attraverso algoritmi o risolutori}. L'output è una soluzione per il nostro modello da implementare nel mondo reale.


% Ingredienti principali
\subsection{Ingredienti principali}

Gli ingredienti principali sanno:

\begin{itemize}
	\item \textbf{dati} del problema
	\item variabili: dette anche var decisionialli: scelte da fare in merito al problema. rappresesntano queindi le scelte, quello su cui il decisore puo internvenire
	\item vioncoli: equazioni che defioniscono i valori che le variabili posson assumere
	\item funzione obietivo: sara una formula che rappresenta una misura di tipo quantitativo per caire quando è buona la soluzione che abbiamo ottenuto. quidni dovremo ottimizzare questo valore in base al contesto
\end{itemize}



parlemeno di programmazione lineare con modeli matematici lineare con delle relazioni lineari dato che ci focalizzaiamo su questi perche molti modelli lineari si trovano in modelli piu complessi e anche perche molti modelli della realta sono lineari e possono essere modellati da essi


DESCRIZIONE PROBLEMA
proviamo a risolvere un problema di mix di produzione, cioe un sistema con un impianto co n2 stabilimenti in cui nel primo diamo le materie prime e vengon orealizzati i componenti in uscita abbiamo i componenti realizzati che arrivano nello stabilimento 2 che e1 quello di assemblaggio e in uscita abbiamo i prodotti finiti

supponendo di voler realizzare 2 prodotti A, B. obiettivo: capire quanti prodotti creare in base ai vincoli e dottenre in guadagno piu alto possibile

avremo ovviamente dei vincoli creati dalle risorse come i macchinari o gli addetti che potranno lavorare un numero di ore finito. supponiamo di avere un guadagno diverso da un pallet di prodotto A rispetto a B


DATI:

ore di lavoro:
		  A		|	B		| addetti
stab 1 | 4 ore	|	2 ore	| 10
stab 2 | 2 ore	|	4 ore	| 10

ogni addetto lavora 40 ore/settimana


profitto in euro/pallet:
A	| B
15k	| 10k

richiesta del prodotto per la prossima settimana:
A	| B
40	| 120


decisione da prendere: determinare il mix di produzione, cioè quante unità di A e B produrre la prossima settimana.

DESCRIZIONE DEL PROBLEMA CON UN MODELLO MATEMATICO:
modellazione:
1. identificare le variabili decisionali:

$x_A$: \# di pallet di prodotto A da realizzare
$x_B$: \# di pallet di prodotto B da realizzare

2. definire la funzione obbiettivo (FO), per massimizzare il profitto

VALORE MAX = $z=15x_A+10x_B$

3. definire i vincoli espressi come uguaglianza o disuguaglianza. in questo caso abbiamo vincoli legati agli stabilimenti

	- vincolo 1: capacità produttiva dello stabilimento 1 $4x_A+2x_B$ che non può superare $40*10$ cioè ore disponibili ogni settimana per un addetto * numero di addetti:
		$4x_A+2x_B <= 400$
	- vincolo 2: capacità produttiva dello stabilimento 2 $2x_A+4x_B$ che non può superare $40*10$ cioè ore disponibili ogni settimana per un addetto * numero di addetti:
		$4x_A+2x_B <= 400$
	- vincolo 3: vincolo sulla richiesta di A: $x_A <= 40$
	- vincolo 4: vincolo sulla richiesta di B: $x_B <= 120$
	- vincolo 5: vincolo di non-negatività: $x_A, x_B >= 0$
	
nella forma completa il modello complessivo è:

MAX: $z=15x_A+10x_B$
sottoposto ai vincoli (sv):
- $4x_A+2x_B <= 400$
- $4x_A+2x_B <= 400$
- $x_A <= 40$
- $x_B <= 120$
- $x_A, x_B >= 0$


RISOLVERE IL MODELLO MATEMATICO (IN GENERE TRAMITE UN RISOLUTORE):
rappresentiamo sul piano tutte le soluzioni ammissibili cercando quella che massimizza il nostro risultato

grafichiamo sul piano cartesiano avendo solo valori positivi ci concentriamo sul primo quadrante

per impostare i primi 2 punti poniamo prina x_A=0 e poi x_B=0. tracciando la retta avremo che al suo interno valgono i vincoli, all'esterno no sarnno quidni punti ammissibili. ed identificano una coppia di punti A e B ammissibili per il primo vincolo. ripetiamo tutto per ogni vincolo chje abbimao

ci verrà fuori una regione ammissibile dove valgono tutti i vincoli dove avremo la nostra soluzione ammissibile. useremo il medoto del gradiente per trovarla:

metodo del gradiente:

vogliamo traovar eil punto che renda massima la nostra funzione z=... 

$\grad{...$ indica la crescita

dove grad z sarà la massima ascesa andando a muoverci in versione concorde al gradiente

tracciando una retta perpendicolare (curve di livello) alla retta del gradiente avremo valori sempre buoni ma più bassi di quelli sul gradiente. facendo attenzione a non uscire dalla regione ammissibile andiamo così a trovare il punto massimo che consente di massimizzare la funzione gradiente e che la segua a che non esce fuori il regime. il punto più estremo alla regione massima sarà il nostro punto

allora la soluzione ottima sarà nel nostro caso quella che fa intersecare le rette del vincolo 2 con il 3: $x_A = 40$ $2*40+4X_B=400$ quindi $x_B=80$

allora la soluzione ottiamle che soddisfa tutti i vincoli ed il profitto massimo possiible e1 in $x_A=40, x_B=80$


notimao che lo stabilimento 2 viene saturato e quello 1 no dato che la soluzione giace sulla retta del vincolo per il quale si satura 




possiamo avere altre forme di modelli di PL
fo da minimizzare
vincoli di ugualinza
vincoli >=
variabili negative
var non vincolate


terminologia:
soluzione: quella di output
sol ammissibile: soluz, se esiste, che soddisfa tutti i vincoli
sol inammissibile: se viola almeno un vincolo 
regione amm: tutti i punti che rispettano i vincoli
prob inammissi: regione amm vuota
prob ammissibile:
	sol ottiuma singola
	ultiple
	fo ilimitata
	
	
% python pulp
in pulp:

\begin{lstlisting}
import pulp as p

# 1. creazione del modello
model = p.LpProblem("ProductMix", p.LpMaximize)

# 2. definisco le variabili decisionall
x_A = p.LpVariable("x_A", cat="Continuous", lowBound=0)
x_B = p.LpVariable("x_B", cat="LpContinuous", lowBound=0)

# 3. definisco la funzione obiettivo in funzione delle variabili decisionali
model += 15 * x_A + 10 * x_B

# 4. definire i vincoli
model += 4 * x_A + 2 * x_B <= 400
model += 2 * x_A + 4 * x_B <= 400
model += x_A <= 40
model += x_B <= 120

# 5. ricolvere il problema
model.solve()

# print della soluzione
print("next week produce {} pallets of A".format(x_A.varValue))
print("next week produce {} pallets of B".format(x_B.varValue))
\end{lstlisting}