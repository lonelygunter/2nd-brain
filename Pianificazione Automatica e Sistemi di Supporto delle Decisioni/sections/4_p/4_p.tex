\newpage
\section{Formulazioni equivalenti di un problema di programmazione lineare}


% Problema in FORMA GENERALE
\subsection{Problema in FORMA GENERALE}

In generale nella \hl{zona ammissibile} diciamo:

$$X={x \in R^n : A_x >= b, D_x = l, x_j >= 0\ \ \ (*)$$ $$\forall\ \ \ j \in J \subseteq \{1,2,...,n\}$$

dove \hl{definiamo la possibilita' di vincoli di $>$=, = e variabili $>$= 0}.

La funzione obiettivo è definita da: $$z = c_x\ \ \ t.c.\ \ \ z = \min z, x \in X$$

che rappresenta il \hl{problema espresso in forma generale}.


% Problema in FORMA CANONICA
\subsection{Problema in FORMA CANONICA}

Se in (*) abbiamo:

\begin{itemize}
	\item $D = 0$
	\item $J = \{1,2,...,n\}$
\end{itemize}

allora il \hl{problema si dice in forma canonica}: $$min_x\ \ \{z = c_x : A_x >= b, x >= 0\}$$


% Problema in FORMA STANDARD
\subsection{Problema in FORMA STANDARD}

Utile quando nella forma generale la matrice $A = 0$ e tutti i vincoli sono espressi sotto forma di uguaglianze:
Se in * A = 0, J = {1,2,..,n}, allora il problema: min_x{z = c_x: D_x = l, x >= 0}

questo problema e1 detto problema in forma standard.



c'e sempre il modo di ricondurci tramite trasformazioni, alla forma standard.



terminologia:

nella forma standard abbiamo che:

z: la funzione obbiettivo (obiettivo travare il min valore per la funzipone z)

A: matrice dei vincoli

D: matrice di dimensione m x n detta matrice dei coefficienti

l: vettore colonna detto vettore dei termini noti

c: vettore di riga detto vettore dei coefficienti di costo (vettore costo)






trasfomazioe da effettuare sul modello per ricondusi alla forma standard:

1. variabili non vincolate di segno: x_j t.c. j non appartiene J

	per trasformarla possiamo sostituire a x_j la somma algebrica di 2 variabili non negative: x_j = x_j ^+ - x_j ^- con x_j ^+ >= 0 e x_j ^- >= 0

	se abbiamo k (<= n) variabili non vincolate in segno , possiamo evitare di introdurre k coppie di variabili non negative. È possibile considerare una variabile x_0 >= 0 e sostituire la generica variabile non vincolata di segno con x_j = x_j ^+ - x_0. Così introduciamo "solo" k+1 variabili



non tutti i voncoli potrebbero essere espressi in forma di ugiaglianza:

2. vincoli del tipo <=
	sum_(j=i) ^n a_(ij)x_j <= b_i
	
	presa una variabile di Slack: S_i >= 0, avremo:
	
	sum_(j=i) ^n a_(ij)x_j + S_i = b_i
	
	
	questa varibile misara lo slack che ce per far si che il vincolo sia rispettato per uguaglianza o no (vincolo > 0)
	
3. vincoli del tipo >=
	sum_(j=i) ^n a_(ij)x_j >= b_i
	
	presa una variabile di Surplus: S_i >= 0, avremo:

	
	sum_(j=i) ^n a_(ij)x_j - S_i = b_i



4. trasformazione di vincoli di uguaglianza in vincoli di disuguaglianza

	basta allora sostituire:
	
	sum_(j=i) ^n a_(ij)x_j = b_i

	sostituiamo a questi:
	
	sum_(j=i) ^n a_(ij)x_j <= b_i
	
	sum_(j=i) ^n a_(ij)x_j >= b_i


5. funzione obiettivo
	se la f.o. è max di z = c_x, è possibile operare la trasformazione:
	
	max z = -min {-z} (meno del minimo di -z)


esempio 1:
min z= x_1 + 2 x_2

vincoli:
6x_1 + 4x_2 <= 24
4x_1 + 8x_2 <= 32
x_2 >= 3
x_1, x_2 >= 0

trasformiamo il problema in forma standard:

se il vincolo e1 in <= allora aggiungiamo una variabile non negativa (slack):

6x_1 + 4x_2 + x_3 = 24

in maniera analoga per il secondo vincolo:

4x_1 + 8x_2 + x_4 = 32

per il terzo vincolo del tipo >= aggiungiamo una variabile ausiliaria non negativa cma la aggiungiamo con un segno negativo:

x_2 - x_5 = 3


vinsolo sulle variabili:

x_1, x_2, x_3, x_4, x_5 >= 0


qjindi nella formalazione stanard avremo j = 1, .. 5 in precedendza avevamo: j = 1, 2



esercizio 2:

trasformare in forma standard il seguente problema:

max z = z_1 + z_2

soggetto ai vincoli:

8x_1 + 6x_2 >= 48
5x_1 + 10x_2 >= 50
13x_1 + 10x_2 <= 130
x_1 >= 0


prima poniamo x_2 = x_2^+ - x_2^-



per trasformala in forma standard:





max z = -min -z = -x_1 -x_2 = -x_1 -x_2^+ + x_2^-


8x_1 + 6x_2^+ - 6x_2^- - x_3 = 48

5x_1 + 10_2^+ - 10x_2^- - x_4 = 50

13x_1 + 10_2^+ - 10x_2^- + x_5 = 130


x_1, _2^+, x_2^-, x_3, x_4, x_5 >= 0



























classe di problemi di ottimizzazione per i problemi complessi:


OPTIMIZATION MMODELS REVIEW:

1. SCHEDULING

reinsta nall'anmito dei problemi dello scheduling dove abbiamo dei task/job gia assegnati. abbiamo a disposizione n macchine/processori. potremmo attrezzare le macchie con dei tools (stumenti per tagliare).


intendiamo allocare i tasks alla macchine in modo overtime quidni caoire anche la facia temportale nella quale eseguire il task.

potrebbe esserci un unico tempo di esecuzione ooppure un task puo avere dei tempi di esecuzioen differenti su macchine differenti. 

L'output sarà un digramma di ganth

i task possono avere degli istanti di rilascio dove un task non posterbbe seesere rilasciato dopo un certo istantedi tempo (ready time)

possoono esserci delle realizioni di precedenza tra i task. quidni non posso effetturar eun task se priam non ho conculso l'altro

i ldiagramam mi dice nel tempo a che macchina è associato quale task ed in quali intervallidi tempo e con quale tool.





nei prblem idi scheduling osn oiporatnti i project scheduling:

2. PROJECT SCHEDULING

progetto: insieme di attività identificate dai task che son orealizzate al fine di raggungere un goal. la acaratteristica di un progetto è che nel complesso le attivita non sono mai state eseguite in precedenza.


lanciare un nuovo prodotto è un progetto dato che e1 un compelsso di attivita mai rilasciate prima

caratterizza un progetto la durata delle attività che non e1 esattamente nota dato che 

tipicamente il progetto e1 a capo del project manager he e1 il responsabile e risponde alla realizzazione del progetto in termini di tempo di realizzazione, costi di produzione, qualità del prodotto.

in genre in project manager usano dei metodi per tener sotto ontrollo il progetto (es: microsoft project) per esempoio degli applicativi che suppostano i p.m. nella schedulaizone e rischedulaizone dei progetti 



i lpiù semplice modello di modellazione:

un porgetto e1 rappresentato da diverse attività:

expected puration




\begin{table}[h!]
	\begin{center}
	\begin{tabular}{|c | c c |} 
			\hline
			Attività & Durata stimata d_i & Predecessori \\ [0.5ex]
			\hline
 			1 & 10 & - \\
			2 & 10 & - \\
			3 & 10 & 1 \\
			4 & 10 & 1, 2 \\
			\hline
		\end{tabular}
	\end{center}
	\caption{Tabella delle ore di lavoro}
	\label{taborelav}
\end{table}


questi problemi vengono rapresntati tramite diagramma aciclico:

rapresntazione activity on node (AoN):

dove abbiamo degli "archi" che rappresentano le predecessioni 

(la tabella mi viene fornita dal project manager dopo aver parlato con delgi esperti)



a volte si introducono dei vertici fittizzi che sono lo start e l'end

start: lo colleghiamo tutte le atticità che non hano predecessori
end: ci colleghiamo tutte le attivtà finali




vedremo come li p.m. può usare delle risorse per accellerare delle attività


prima cosa: cariabili decisionali: nel nostro caso è lo start time. ipotizziamo che i lprogetto inizi al tempo t = 0, quidni per ogni task abbiamo che s_i >= 0 \forall i \in TASKS dove s_i è lo staart time del task i che appartiene a TASKS

in più possiamo definire T >= 0 che è il tempo di completamento del progetto (complition time)


in questo modello vogliamo minimizzare il complation time : min z = T

cioè: z = 1T + 0s_1 + 0s_2 + ...sucamiento


per quanto riguarda le relaizoni di precedenza:

p_{ij} = { (sistema) 1 <=> i è predecessore di j, 0 altrimenti

con p_{ij} una matrice costate e binaria:

p = 

0 0 1 1
0 0 0 1
0 0 0 0
0 0 0 0




VINCOLI DI PRECEDENZA:

T è un maggiorante del tempo di completamento delle task, quindi:

s_i + d_i <= T

p_j (s_i + d_i) <= s_j \forall i \in TASKS, j \in TASKS
se p_ij = 1 allora i e1 predecessore di j quindi il tempo di dinizion del task j deve sessere duccessivo o uguale al task i cioe1 s_i + d_i


se invece i non e1 predecessore: p_ij = 0 quindi avremo che 0 <= s_j allora il vincolo è ridondante dato che e1 un dato che già abbiamo


questo vale per un generico numero di task, mi serve pero un lingiaggi odi modellaizone per scrivere il modello in forma compatta per avere dati e modello divisi


per semplifia l'ultimo vicolo possoiamo scrivere 
s_i + d_i <= s_j \forall i \in TASKS, j \in TASKS, p_{ij} >= 0





esempio modello espanso per problemi di istanza con i dati di sopra:

min z = T

s_1 + 10 <= T
s_2 + 10 <= T
s_3 + 10 <= T
s_4 + 10 <= T

s_1 + 10 <= s_3 (p_{13} = 1)
s_1 + 10 <= s_4 (p_{14} = 1)
s_2 + 10 <= s_4 (p_{24} = 1)

s_1, s_2, s_3, s_4 >= 0
T >= 0



