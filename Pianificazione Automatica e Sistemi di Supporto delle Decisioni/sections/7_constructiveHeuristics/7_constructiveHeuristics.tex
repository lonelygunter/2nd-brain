\newpage
\section{Constructive heuristics}

% Introduzione
\subsection{Introduzione}
Nel (caso peggiore) di variabili binarie il livello computazionale sarà dato da $O(2^n)$ quindi il tempo di calcolo cresce esponenzialmente con il numero di variabili. 

In molte applicazioni si fa affidamento ad \hl{algoritmi detti inesatti} che \hl{CERCANO di generare delle soluzioni ammissibili}. Questi rientrano negli \hl{algoritmi euristici} dove è importante che la \hl{conoscenza del progettista venga trasmessa all'algoritmo}. Questi algoritmi possono essere:

\begin{itemize}
    \item \textbf{costruttivi}: dove \hl{cercano di generare una prima soluzione ammisibile}
    \item \textbf{migliorativi}: dove \hl{prova a migliorare la prima soluzione}
\end{itemize}

Trovare una soluzione ammissibile potrebbe essere difficile dato che potrebbe essere più complicato del trovarne una ottima.


% Travellig Salesman Problem (TSP)
\subsection{Travellig Salesman Problem (TSP)}

Data una matrice $C$ di transizione dove, per andare dal punto $1$ al $2$ \hl{pago $c_{12}$} e così via per tutte le possiabili iterazioni \hl{per un numero $n$ di punti da "visitare"}. Lo scopo sarebbe trovare un circuito che tocchi tutti i punti una sola volta con \hl{costo minimo}.

Avremo che il \hl{tempo di ciclo determina la produttivita' della macchina} tipo un robot che fa n fori e conclusi gli n fori finisce oil cilo.

Un vincolo ulteriore potrebbero essere le \hl{finestre temporali} che potrebbero esserci come nei casi di consegna dei pacchi amazon, il che \hl{rende computazionalmente piu' difficile o anche inammissibile il problema}. Notare come cambiando anche solo un dato potremmo avere una soluzione ammsisibile o una crescita esposnenziale dell'infattibilità dala quele deriva l'instabilità.


% Algoritmo greedy
\subsection{Algoritmo Greedy}

cerca di massimizzare nel breve periodo. procedura sequenzaiale ch costruisce la solzuone un  passo alla volta 
è euristica di tipo costruttivo costruiendo la soluzione con un tipo sequenzizale ad ogni passo fa una scelta e massimizza solo l’utilizzo immediato ma cosi puo capirea che
- ho sluzione inammissibile
- non garantisce una soluzione ottima

es: 

pseudocodice greedy o anche detto nearest neighbour: adattamenteo dell’algoritmok greedy al commesso viaggiatore

parto dal punto 1 e lo identifico cole last cioe ultimo punto tocacto 
S={…}e poi una lista di punti da visitare

while non ho toccato tutti i punto S != insieme vuoto
estraggo un punto da S
cerco il punto a costo minimo rispetto al last 
storicizzo il valore del last succesivo
imposto il last alnuovo punto 
end while
riporto il succ last = 1 per poterlo riportare all’inizio


il vantaggio è che possiamo prevedere le iterazioni del loop 

es:
n = 4
C = 
0 10 5 8
10 0 2 1 
5 2 0 4
8 1 4 0

last = 1 rho = <2,3,4>
last = 3 ; rho=<2,4>; succ_1 = 3
last = 2; rho=<4>; succ_3 = 2
last = 4; rho =<>; succ_2 =4
succ_4 = 1

ovviamente se abbiamo delle time window l’algoritmo deve essser adattato. quento greedy infati serve per poter essere customizzato per i problemi




classkwork

tramite TSP

usiamo libreira pulp


modello a varaibili binare x_{ij} per ogni coppia di punti i e j ed è 
- =1 se l’arco x{ij} è visistato
- =0 se non è viositato

quindi orendiamo le x_{ij} che ci interessano e le imporiamo ad 1 e le altre a 0 dato che queste non fanno parte del nostro “percorso”

img

vogliamo usare la sommatoria {della f.o. }dei costi  dove voglio minimizzarli quindi sostituiamo le x con quelle sopra e le c in base alla tabella dei costi data

secnda somm quindi ogni punto aha un successore
terza sum ogni punto ha un precedessore

quesnti vincoli non basta no per poter risolvere il modello infatti non possiamo esculudere che la soluzione sia disconnessa (sub-tour)


img


dato che mancano dei vincoli di connessione della soluzione


siamo quindi il modello MTZ: miller-zucker0zemlin:
avremo delle variabili addizzionaliu_i una per ogni punto e conu_i rappresenta l’oridine di visitade punto i

imporiamo pertendo dal pnto 1
u_i = 1

abbiamo allora ù =1 -> x_13 = 1 -> u3 = 2 -> x34 = 1 -> u4 = 3 -> x42 = 1 -> u2 = 4 -> x21 = 1

allora agigungo quenste variabili ui per rappresentare l’ordine di vista e non avere disconitnuita della soluzione

altri vincoli saranno alora:
2 <= ui <= n

dato che le ui con i != 1 saranno comprese tra 2 a n dato che dobbiamo assegnare la 2 posizione finio alla n-esima

poi una ltro vincolo:
ui - uj + 1 <= n(1-xij)

lega le variabili u con xji che dicono cihi viene dopo chi e u l’ordine. questa relaione è corretta dato che abbiamo 2 casi:”
- se xij = 0 allora ui - uj +1 <= n ovvio dato che tutte le variabili u son ocomprese in n 
- se xij = 1 allora ui - uj + 1 <= 0 dato che uj >= ui + 1 allora i è predecessore di j  alora l;odine di visita di j è almeno successivo a i


altro paradigma che si adata bene ad altre classi di problemi:
relax and fix algorithm

sono problemi multiperiodali dato che voglio poter avere un approccio ora per ora. uso un idea algortimo dvoe suppongo di voler risolvere ik prob di ottimizzazione:
min z = ..
s.a.
Ax = b
x >= 0 e intera

alora vado a partizionare le variaibli in n gruppi. quijdi un in problema multi periodiale di pianificazione da quoi a un tempo t, le variabili si dividono in gruppi, decisioniali dche mi dicocosa fare qui alle 12, quelle delale 12 alle 13 eccecc.

abbiamo alloradelle applicazioni ndove le varaibili naturalemte si dividono in gruppi datoun osviluppo temporale, il orblema allora diventa:

min z = sum..
s.a.
sum...

il prblema nel complsso è complicato, allorami rendo conto che la diffficolta sta nelle molte varibili intere, alllora definisco un oroblema aursiliario dove le variaibli x_1 del primo gurppo , sono intere mentre le restatni faccio un rilassamento, quindi i l problema asuliairio è guale all'originale ma solo le x_1 sono intere e le altre rilasate. queto problema è ancora a varaibili intere, pero è piu semplice del prob di partenza dato che solo le var x_1 sono intere.

risolto questo peblema, se è inammissibile, acnhe quello di partenza lo è, se però ammette soluzione ottima, allora avro che x_1 è intera e faccio un fixing quindi nell iterazioni successvie fosso x_1 al valore che mi è veniuto nella risoluzioen del problema 
x_1 = \overline{x}_1

alliterazione successiva il vincolo di x fa parte del problema e uindi x_1 è fissato al valore ottenuto prima, a x_2 richiedo di essere intera e le variaibli dei gruppi successivi sono ancora rilassate se il promeba rislto è inamissibile, fallisce. se ha successo, risolvendo troviamo \oveline\overline{x}_2 facciamo un fix fissando x_2 = \oveline\overline{x}_2

tengo quindi conto anche delle variabili "future" ma non richiede un grande sforzo computazione dato che labvora sul singolo gruppo.

la genereica iterazione è:
....

pseudocodice:
.....



approcci orolling horizon:
dyrante il primo periodo considero un sottoproblema con alcune varaibili da x_1 a x_k e transcuro quelle dei periodi successivi, considero alalora un sottproblema con solo ele varaibili x_1 fino x_k e faccio un fix su x_1 e faccio un passo in avanti, quindi considero un sottoproblem con x_1 fissato e considero le variabili x_2 fino a x_k+1.

un porblem acon k varaibiali riwco a risolverlo.

quidn risolvo dedei sottoporbemi con variavili legate con tempi consecutivi ma non quelli troppo successivi

(notare che è meno accurato del relax and fix dato che considero certe varaibili e non tutte assieme)





MULTIPLE CRITERIA DECISION MAKING:
dato che spesso abbiamo piu obbiettivi non riusciamo a scegliere in anticipo quale ha rpiorita allora spesso vanno in conflitto, allora per capire come fare. 

usiamo la materemati multecreatirea dove abbmiamo da trovare una soluzione con le migliore soluzioni che devon orispettare alcuni vincoli, pero abbiamo un numero p cioè il creiterio. quidni dobbiamo massimizzazare o min non una solo af.o. ma da z_1 a z_p.

es: problema si schediling:
abiamo n attivita e come paramentri:
S_i starting time dell'attivita i
d_i durata dell'attivita i
pij = 1 se l'attivta j è prerequinistito dell'attivita i, =0 altrimenti

obiettivo: minimizare il tempo di completamento = T


avremo allora che:
min T
s.t.
T >= S_i + d_i ....



formulare un attivta in modo fattiibile:
D_i max durata dell'attivita i
b_i min durata dell'attivita i
d_i durata dell'attivita i
X_i costo per accellerare il comletamento dell'attivita i

allora:
d_i = D_i-a_iX_i con a_i coefficente di ...
...

allora bbiamo da minimizzare 2 obiettivi:
min \sum .....
min T
s.t.
.....


considerato un oroibkem acon piu obiettvi allor apossimo consederaene uno con obiettivo singolo. quindi le soluzioni ottimali dei singoli porblemi per rappresentarle scrivo nello spazio ammissibile, lo spazio dei criteri rappresentando i singoli obiettivi. a volte puo succedere che presi i du epunto la loro unione non fa parte della zona ammissibile (utopia)

img

ma noi allora adiamo a prendere il punto migiore nella zona ammissibile come in questo caso C dato che è migliore di A per certe cose e per altre rispetto a B

potrmo prender per o un punto D che saràpegio re rispetto ad entrambi gil obiettivi dato che son osoluiozni dominate dareo che f1(D) >= f1(C), e f2(D) >= f2(C)

a noi interesse di trovare le soluzioni non inferioir e quindi efficienti e dobbiamo traovare tra tutte la migliore. per capire qual è quella migliore dipende dal 


si usa allor aun trade off ratio dove prendiamo le soluzioni A e B e calcolipmo il rapporto di trade off facendo:
....

che rappresenta il miglioramento dell'obiettivo iesimo in base al miglioramentodi un altro obiettivo ?!

noi allora ci interessiamok delle soluzioni sulle frontiere dato ceh quel einterne sono dominate da altre piu vicine alla frontiera

se abbiamo una zona convessa avremo che si collegherà in modo immaginario i punti togliendo la convessità e quindi prendiamo i punti sulla forntiera tranne sulla parte immaginaria


per trovare le soluzioni effcienti abbiamo 2 metodi:
- metodo dei vincoli-
- metodo dei pesi


1. prendiamo iun singolo obiettivp e per gli altri imponiamo una soglia dove per ogni obbiettco diverso da quello scelto la funzione obiettivo si a al amssismo:
fk(x) <= uk (dato che minimizziamo)
.....

img


es: nel caso di multischeduling:
vogli ominimizzare il costo come obiettivo, impongo poi un valore soglia al tempo di compeltamento:
T <= T^*

con T^* tra Tmin e Tmax in base al fatto che se vogliamo minimizzare o massimizzare



altro metodo: dei pesi:
abbimao una sola f.o. come somma pesate degli altri obiettv:
\sum wi fi

com wi obiettivo: 'sum wi = 1

a seconda del valore che diamo ai pesi andiamo a sposatre la retta trangente alla zona ammissibile

img

questo non funziona se è convesso dato che non possiamo traovare tutte le possibili soluzioni dato che la forntiera non è connessa. 

es: con multiobeittvio:
min.....





per scegliere tra le diverse soluzione e traovare quella migliore, dipende dalle preferenze che posson essere espresse tramite una funzione che a seconda de valore dei diversiobiettivi assume un valore di verso allora traovando la suozlione che massimizza il valore della funzioen traviamo al soluzione migliroe per il decisore, il problemaè che in gere non c'è una sluzione matematica per questa cosa ma  posiiamo:
max U ....
s.t.
g(x) >= 0
x >= 0

metodo a priori
se disponiamo di questa funzione in forma chiusa possiamo risolvere i lproblema monoobeittivo masismizzando la funzione (detto metodo a propri)

in genre ono si trova questa solzione

es: settiamo una soglia erapprensetiamko la distaza di f.o. e la soglia adeguata. al variare di r avremo  3 tipi di metriche:
- r=1 ....
- r=2 metrica euclidea ....
- r=+infty .....

sara difficile di re per ogni obeittvi oquale puo esser una solgia adeguata dato che spesso traovare singoli ottimi è complicato.




altro apporccio: a posteriori
andiamoa a genreare tutte le soluzioni effcienti e dire quale tra le soluzione è la migliore. è difficiel applicare al metodo perchè potrebbe cresce in maniera espoenziale con la dimensione del problema



allora usiamo un approccio interattivo:
approccio di mezzo
presi 2 obiettvi troviamo il punto ottimko per entrambi e poi troviamo la priam soluzioneefficiente che si atra A e B e poi si scegli a su quale parte della forntiera vogliamo posizionia=arci. es parte AB e trbiamo una oluizone efficente tra le 2 ecc ecc
 
img

mi fermo quando il segmento che mi rimane è cosi piccolo che i due punti con=incidono

m isposto in base a cosa mi server