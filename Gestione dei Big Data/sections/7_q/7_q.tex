\newpage
\section{Distributed Database Concepts}

% Distributed Databases
\subsection{Distributed Databases}

I dati utilizzati nelle infrastrutture dei big data \hl{devono essere ACID}. Queste infrastrutture sono \hl{composte da nodi che collaborano per compiere un task}. In queste infrastrutture andremo a distribuire le risorse sui nodi che cooperano in modo da avere \hl{ridondanza di dati}.

Esiste una \hl{relazione logica tra questi database connessi}, ma non tutti i nodi devono essere omogenei quindi possiamo al concetto di \hl{DISTRIBUTED DBMS} che deve gestire l'avere modelli dati connessi. 

Per quanto riguarda le \hl{query} bisognerà riorganizzarle per gestiore nodi distribuiti.

abbiamo varie forme di trasparenza rispetto all'utente:

per organizzazione dei dati:
- location transparency ?
- naming transparency: dove lo sviluppatore ha una lista

per repliche: usate per ridurre la mancanza del servizio

per frammentazione dei dati:
- partizione orizzontale: abbiamo la stessa struttura dati ma i dati salvati sono diversi abbiamo una partizione delle tuple
- partizione verticale: frammentazione del modello dati (possiamo avere tabelle con attributi su un nodo ed altri su un altro nodo)

es partizione orizzontale dato che provengono dalla stessa tabella:
img

come faccio a capire che i dati siano esattamente tutti quelli che mi servono?

cerco di rispettare affidabilità e disponibilità: tramite la ridondanza

distribuiamo per poter avere la possibilita di scalare i dati in modo orizz o vert a seconda dell'uso che dobbiamo farne. parliamo di:
- partition tollerant: il sistema è capace di operare anche se è partizionato

il cap theorem dice che se partizionato non si può garantire consistenza e disponibilità.

caratteristiche della distribuizione è che ogni nodo viene ad essere autonomo quindi per fare la progettazione dei nodi, la porgettazione deve essere tale che ciò ceh c'è sui nodi deve essere autonomo andadno a guarare:

- design autonomy: ciò che l'uso dei modelli deve permettere di gestire delle transazioni in imodo efficiente
- communication autonomy: deve dare la possibilità di far condividere le informazioni
- execution autonomy: non son osicuro quando faccio la query di come si afatta ma che laposso fare deve essere assicurato


vantaggi delle architetture distribuite: migliora la facilità di sviluppo, la disponibilità, le performance


frammenti: unità logiche dei database

framm orizz: divide le relazioni in orizzontale (tuple)
framm verticale: divide le relazioni in verticale (colonne)

frammentazione orizontale completa: se tramite union possiamo ricostruire la nostra lezione completamente
frammentazione verticale completa: usiamo la join

spesso abiamo una frammentazione sia verticale che orizzontale e dovremo andare a ricorstruire il db, parliamo allora di frammentazione dello schema.

un DDBMS sarà in grado di gestire una frammentazione usando un ALLOCATION SCHEMA che dice come i frammenti del db son odistribuiti su quali nodi. quando abbiam oi dati può succedere che per ragio di backup replichiamo il db(non la soluzione migliore). invece posso avere delle repliche a caldo che ogni tempo t replicano su un altro sito una perte dei dato che son ostati modificati, per questo uso un database journal che è un catalogo che tiene traccia di tutti i db dove trviamo i log di tutti gli eventi accaduti sui db poteendo poi ricavare tutte le info sulle modifche ecc per quanto riguarda la repllica. e poi se i dati non oriescono ad essere accessibili posso recuperare l'ultima operazione buona. le repliche son oprogrammate

possiamo realizzare una replica totale o perziale, replicando sia la struttra che i dati a secondo dello schema di replica presente del DDBMS

VEDERE CAP 20-21-23

problemi per DDBMS:

- copie multiple dei dati
- two phase commit (uno parla con un altro e gli viene restituito un ack -1 se non è andato a buon fine il passaggi odi dati per aggiornare il db)

esistono vari modi per gestire le copie: ci puo1 esere un sito primario ed altri secondari che posson essere il suo backup. quindi andrò a gestire le tabelle in modo parallelo oppure se facio una copia di un sito e non ho un backup in tempo reale. se ho piu siti con le repliche posso avere il problema che se il sito primario fallisce dovrò elegere un altro sito come quello successore del primario.

uno dei modi per gestire il nuov o responsabile del locking è con un sistema a voto dove si amndano delle richieste a tutti e si vota, ogniuno mantiene i lloro pezzo bloccato, e dopo la transazione si definisce che puo1 esere eletto come responsabile. se non si riesce ad avere in un certo tempo si usa un timeout