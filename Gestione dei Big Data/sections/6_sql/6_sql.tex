\newpage
\section{Basic SQL}


SQL è un linguaggio che consente di accedere al db in varie modalità ed \hl{ha la funzione di creare e gestire i db}.


% Statement - SELECT
\subsection{Statement - SELECT}

Usato per \hl{recuperare informazioni dal db}. la sua struttura ha 3 clausole (claude):

\begin{lstlisting}
SELECT <attribute list>
FROM <table list>
[ WHERE <condition> ]
[ ORDER BY <attribute list> ];
\end{lstlisting}


Molto utile usare gli \hl{alias (AS)} per:

\begin{itemize}
	\item andare a \textbf{definire i campi che ci serviranno in modo da dividere gli attributi di una tabella con quelli di un altra}
	
	\item \textbf{accedere ad una stessa tabella ma con 2 alias diversi} perchè per esempio uno rappresenta l'impiegato e l'altro il supervisore
	
	\item \textbf{rinominare gli attributi}:

\begin{lstlisting}
EMPLOYEE AS E(Fn, Mi, ...)
\end{lstlisting}


\end{itemize}


\hl{Keyword} da poter usare:

\begin{itemize}
	\item \textbf{DISTINCT}: restituisce solo valori distinti (diversi) nel set di risultati
\end{itemize}


% Statement - WHERE
\subsection{Statement - WHERE}

\hl{Esprime una condizione}, se manca è possibile fare il prodotto cartesiano se si usa:

\begin{lstlisting}
SELECT Ssn, Dname
FROM EMPLOYEE, DEPARTMENT
\end{lstlisting}

Si possono usare delle condizioni di tipo:

\begin{itemize}
	\item numerico:

\begin{lstlisting}
WHERE Dno = 5
\end{lstlisting}

	\item pattern matching tra stringhe:

\begin{lstlisting}
WHERE Ssn LIKE "yes"
\end{lstlisting}
	
		se non è un occorrenza esatta usiamo:

			\begin{itemize}
				\item \textbf{\%}: indica una qualsiasi sottostringa
				\item \textbf{\_}: indica un solo carattere in una specifica posizione
			\end{itemize}

\begin{lstlisting}
WHERE column IN (SELECT Statement)
\end{lstlisting}

vado a selezionare da Statement gli attributi column


\begin{lstlisting}
WHERE column BETWEEN value1 AND value2
\end{lstlisting}

vado a selezionare i valori tra value1 e value2 gli attributi column

\end{itemize}


% Statement - ORDER BY
\subsection{Statement - ORDER BY}

Per ordinare i risultati con DESC o ASC.


% Statement - INSERT
\subsection{Statement - INSERT}

Usata per inserire dei nuovi dati nei DB. 


% Statement - GROUP BY
\subsection{Statement - GROUP BY}

Usato per raggruppare uno o più attributi in base ad un certo valore.

\begin{lstlisting}
SELECT COUNT(CustomerID), Country
FROM Customers
GROUP BY Country;
\end{lstlisting}


% Statement - HAVING
\subsection{Statement - HAVING}

Usata come un WHERE, quindi filtro, ma usando delle funzioni aggregate

\begin{lstlisting}
SELECT COUNT(CustomerID), Country
FROM Customers
GROUP BY Country
HAVING COUNT(CustomerID) > 5;
\end{lstlisting}


% Statement - UNION
\subsection{Statement - UNION}

Usato per combinare il risultato di 2 o più statement SELECT

\begin{lstlisting}
SELECT City FROM Customers
UNION
SELECT City FROM Suppliers
ORDER BY City;
\end{lstlisting}


% Statement - JOIN
\subsection{Statement - JOIN}

Usata per unire più risultati di operazioni disseminate tra più tabelle.

\begin{lstlisting}
SELECT Orders.OrderID, Customers.CustomerName, Orders.OrderDate
FROM Orders
JOIN Customers
ON Orders.CustomerID=Customers.CustomerID;
\end{lstlisting}


% Values - NULL
\subsection{Values - NULL}

Rappresenta un valore nullo che può dare problemi nel caso di machine learning o raccoglimento di dati.

Usandolo come statement:

\begin{lstlisting}
SELECT column_names
FROM table_name
WHERE column_name IS NULL;
\end{lstlisting}


% Function - Aggregate
\subsection{Function - Aggregate}

Eseguono un operazioe su n valori dati e possono essere:

\begin{itemize}
	\item COUNT
	\item AVG
	\item SUM
	\item MAX
	\item MIN
\end{itemize}

\begin{lstlisting}
SELECT COUNT(column_name)
FROM table_name
WHERE condition;
\end{lstlisting}


% Function - Window
\subsection{Function - Window}

\begin{itemize}
	\item LAG
	\item LEAD
\end{itemize}


% Function - String
\subsection{Function - String}

\begin{itemize}
	\item CONCAT
	\item LEN
	\item UPPER
	\item LOWER
\end{itemize}