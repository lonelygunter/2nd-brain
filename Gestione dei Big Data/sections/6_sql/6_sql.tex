\newpage
\section{Basic SQL}


SQL è un linguaggio che consente di accedere al db in varie modalità ed \hl{ha la funzione di creare e gestire i db}.


% Statement - SELECT
\subsection{Statement - SELECT}

Usato per \hl{recuperare informazioni dal db}. la sua struttura ha 3 clausole (claude):

\begin{lstlisting}
SELECT <attribute list>
FROM <table list>
[ WHERE <condition> ]
[ ORDER BY <attribute list> ];
\end{lstlisting}


Molto utile usare gli \hl{alias (AS)} per:

\begin{itemize}
	\item andare a \textbf{definire i campi che ci serviranno in modo da dividere gli attributi di una tabella con quelli di un altra}
	
	\item \textbf{accedere ad una stessa tabella ma con 2 alias diversi} perchè per esempio uno rappresenta l'impiegato e l'altro il supervisore
	
	\item \textbf{rinominare gli attributi}:

\begin{lstlisting}
EMPLOYEE AS E(Fn, Mi, ...)
\end{lstlisting}


\end{itemize}


\hl{Keyword} da poter usare:

\begin{itemize}
	\item \textbf{DISTINCT}: restituisce solo valori distinti (diversi) nel set di risultati
\end{itemize}


% Statement - WHERE
\subsection{Statement - WHERE}

\hl{Esprime una condizione}, se manca è possibile fare il prodotto cartesiano se si usa:

\begin{lstlisting}
SELECT Ssn, Dname
FROM EMPLOYEE, DEPARTMENT
\end{lstlisting}

Si possono usare delle condizioni di tipo:

\begin{itemize}
	\item numerico:

\begin{lstlisting}
WHERE Dno = 5
\end{lstlisting}

	\item pattern matching tra stringhe:

\begin{lstlisting}
WHERE Ssn LIKE "yes"
\end{lstlisting}
	
		se non è un occorrenza esatta usiamo:

			\begin{itemize}
				\item \textbf{\%}: indica una qualsiasi sottostringa
				\item \textbf{\_}: indica un solo carattere in una specifica posizione
			\end{itemize}

\end{itemize}


% Statement - ORDER BY
\subsection{Statement - ORDER BY}

Per ordinare i risultati con DESC o ASC.

