
\section{28.09.22}

RIEPILOGO DI CONCETTI DI BASE DEI DB RELAZIONALI:

cap 1: database
1. definizioni di base:
dato: insieme di fatti conosciuti registrati con un significato. Sono detti dati grezzo visto che si suppone che andrò a elaborarlo, questo dato sarà poi archiviato, sarà un fatto conosciuto cioè degli eventi con un significato per un dato tipologia di utenti ed e1 conosciuto dato che c'e1 una sorgente che produce i dati con una cerca velocità.

database: raccolta di dati altamente organizzati intercorrelati e strutturati. e1 una struttura con dei collegamenti strutturati tra i dati

dbms: data base managment system: insieme di programmi per accedere ai dati e farci delle operazioni di 4 tipi: creazione, recupero, aggiornamento e cancellazione, ciclo CRUD.

esistono molti tipi di db i primi erano solo numerici o testuali, ora ci sno nquelli multimediali, GIS Geographic Information Systems, data warehouses

opportuno vedere dei concetti di base. i dati hanno un ciclo di vita, il piu1 semplic è:
acquisizione (scattered data) -> aggragazione (integrated data) -> alisi (knowledge) -> finisce in un apllicaizone che genera dei log data che sarannp poi acquisiti come scattered data

da u punto di vista computazionele queste fasi si devono prendere in un altro modo:
1. storage dei data
2. formattazione e pulizia
3. capire cosa i dati ci dicono 
3.? se non mi bastano i dati che ho posso integrare dei dati 

definizioni:
database: collezione di dati collegati tra loro

mini-world: parte del mondo real ai quali si riferiscjno i dati presi. per creare un database vado a limitare la modellazione in un numero n di concetti

dbms: sistema che afacilita mantenimento e gestiondel db

database system: insieme di dbms con i dati


quando si ha un db abbiamo 3 livelli da considerare
fisico (dove sono salvati
logico: come sono collegati tra loro
vista: rappresentazione che sarà diversa per ogni tipo di utente


da un punto di vista architetturale da un aparte ho gli utenti a dall'altra i dati e i metadati (che descrivono i dati). avrò bisognodi un software per eseguire le query 

un dbms offre:
di db mi consente di fare più del semplice slvataggio:
definire modelli dati
manipolarli
processarli e condividerli

per interagire coi db possono esere tramite:
query: accede a parti differenti di dati e formula una richiesta
transazioni:legge dei dati ed aggiorna alcuni valori e li salva nel db


nei nostri mini-world avremo bisogno di identificare delle entità cioè i concetti di base che rappresentano una parte delle cose che inseriremo nel db relaizonele il resto sta nel connettere tra loro le entità, dette relazioni (relationships) (ER) le tabelle che ne derivano sono dette relation. tutto cio deve derivare dai requisiti e non dal'esperienza personale.

le entità diventano delle tabelle dove inseirisco i dati che ho a disposizione che saranno divise per righe (record) e colonne (attributi) i singoli elementi sono i dati grezzi

catalogo dei dbms ci sono i vincoli creati da un software specifico, i ltipo dei dati e la relaizone di appartenenza degli attrbuti

concetto fondamentale: astrazione:
ho dei dati e collegamenti tra varie entità e dei modi per salvrli. dobbiamo separare le cose per disporre di un modello senza che esso si occupi di come salvare i dati

modello concettuale:

modello fisico: definiziaone dei tipi dato e dove sono conservati

controllo della concorrenza: garantire ceh tuttle le transazioni sono correttamente eseguite
recovery: se la transazione è stat eseguita e1 stata recorded nel database




cap 2: 


data model: insieme di concetti che descrivon ola struttura di un db, le operaizoni e i vincoli applicati al db

data model structure e constraints: abbiamo dei costrutti che definiscono come collegare gli elemtni definiti da: entità, record e tabella
i vincoli devono essere sempre rispettati e bloccanti

data model operation: di base (CRUD) o definite dall'utente

modello dato di tipo concettuale: di alto livello e semantico

mkodello fisico: definisce come i dati sono salvati, basso livello

modello tipo implementativo: usati nel dbms 

modello implementation: provide concepts taht fall between the above two

modello autodescrivente: basati su XML 



3 definizioni fondamentali:
database schema: descrizione del database in termini di sctuttura tipo dati e vincoli

schema diagram: visione rappresenzativa del db schema

schema construct: insieme tra schema ...


figura 2.1 schema diagram

database state: smapschot in istante t del db, si devfinisce quidni ai suoi contenuti

valid state: si definisce funzionante se il suo contenuto soddifa i vincoli per quello schema

nota che:
schema può esere detto intensio
state può essere detto extension


abbiamo 3 livelli di schema:
interno (fisico): come i dati devono essere salvati e come posso accederci
concettuale
esterno: per descrivere le view dell'utente


per passare d aun o schema ad un ator ho bisogno di un mapping per capire a cosa corrsponde un elemento. avremo che:

logic data independence: se voglio cambiare lo schema concettuale senxa cambiar equello fisico

physiccal; devo cambiare lo schema sifico senza cambiare quello concettuale



qundo ho un data manipulation language dml posson usarlo inautonomia o inserirlo in un orogramma facendop tramite:
embedded approsh (tramite java, c ...)
procedure call approach
db programming language approach
scripting languages


definizione
data dictioraty: insime dove salvo sia lo schema che altre info 



abbiamo più tipologi e di dbms:
centralized: dfove abbiamo tutta l'eleborazioe su un unico nodo

2-tier: sistpecializza in termini di server per ogni blocco di funzionalità che devo offrire

cliets: per far accedere gli utenti

dbms server: per esegjuire query e transazioni tramite API 

three-tier: abiamo 3 livelli: ...


