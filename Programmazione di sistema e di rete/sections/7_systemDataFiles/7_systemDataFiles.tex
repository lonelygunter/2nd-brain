\newpage
\section{System Data Files and Information}


% Password file
\subsection{Password file}

Nei sistemi Linux è presente come: 

\begin{lstlisting}
/etc/passwd
\end{lstlisting}

Nei sistemi MacOS, invece, non può essere vista in modalità "normale", infatti si può vedere solo se in modalità single user.


% Time and Date Routines
\subsection{Time and Date Routines}

Sono delle \hl{routine per effettuare il conto del tempo utile} per il profiling di un programma dove si va a capire quando il programma si è tardato con l'esecuzione:

\begin{itemize}
    \item \hl{time()}: che ritorna il tempo in sec facendo riferimento alla epoch
    \item \hl{gettimeofday()}: aggiunge i microsec quindi abbiamo una struttura di 2 interi
    \item \hl{clock\_gettime()}: aggiunge i nanosec quindi abbiamo una struttura di 2 interi
\end{itemize}

Questo tempo può essere \hl{trasformato in una data}, infatti possiamo andare dai sec ad una struttura data tramite:

\begin{itemize}
    \item \hl{gmttime()}: tempo del meridiano 0
    \item \hl{localtime()}: info sulla timezone raggruppate nella struttura "tm"
    \item \hl{mktime()}: per effettuare il processo opposto
\end{itemize}  

Per vedere quando compiremo 1.000.000.000 di sec:
\begin{lstlisting}
date -r $(($(date -j -f %Y%m%d-%H%M%S 20010101-190000  +%s) + 1000000000)) '+%m/%d/%Y:%H:%M:%S'
\end{lstlisting}
