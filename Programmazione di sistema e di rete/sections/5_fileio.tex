\newpage
\section{File I/O}


% Introduzione
\subsection{Chiamata open}

I file I/O sono le \hl{funzioni che gestisce buffered I/O} ed in contrapposizione da quelle della librerira "stdlib". La chiamata open() fa parte di queste funzioni, i suoi argomenti sono:

\begin{itemize}
	\item \hl{arg}: path assoluto o relativo
	\item \hl{flag}: bit che indica l'\texybf{attivazione di alcune modalita'}. Si avrà allora a \textbf{settare il bit della flag a 1}:
	\item \hl{mode}: serve a dare i \textbf{privilegi con cui i file deve essere creato} (\textbf{da usare solo nella creazione del file})
\begin{lstlisting}
open(file, O_RDWR | O_APPEND | O_CREAT | O_TRUNC, file_mode)
\end{lstlisting}

		avremo allora: $11000001010$

		con O\_RDWR: 2, O\_APPEND: 8, O\_CREAT: 512, O\_TRUNC: 1024
		
		
\end{itemize}

Appena \hl{aperto il file questo avra' la "current position" a 0}, cioè ad inizio file. Iniziando a scrivere avremo che alla prima lettura la nostra current position si sarà spostata fino a dopo abbiamo finito di scrivere.

Per quanto riguarda \hl{read e write i bit delle flag ....}

Una \hl{variante e' openat()} che \hl{prende il file descriptor (passato dalla open() sulla directory) di una directory per passare il path relativo a quella directory}. Abbiamo quindi la \hl{possibilita' di accedere a delle direcotry anceh senza avere i privilegi}. ???
