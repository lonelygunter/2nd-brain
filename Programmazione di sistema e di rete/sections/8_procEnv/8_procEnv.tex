
cap 7:
process environment:

per terminare un porcesso usiamo:

\begin{enumerate}
    \item return dal main
    \item chiamando exit: chiudo la singola thread ma poi passa da un gestore di exit tramite una exit function che chiama deli exit handler che posson essere installati e aggiunti tramite la funzione
\begin{lstlisting}
int atexit(void (*func)(void));
\end{lstlisting}
    e verranno richiamate in ordine contrario alla dichiarazione
    
    \item chiamando \_exit oppure \_Exit: usate per terminare il processo e finire direttamente nel kernel senza passare dall'exit function
    \item return dell'ultima thread
    \item chiando pthread\_exit dall'ultima thread
    \item chiamando abort (che genera un segnale che può essere gestito SIGABRT, ma fa finire comunque il processo)
    \item ricevendo un segnale: come SIGKILL e SIGSTOP che non potranno essere gestiti o fermati
    \item l'ultima thread riceve una richiesta di cancellazione
\end{enumerate}



Environment list:
uando si lancia a=100 nello schell e poi ne lancio un altrola variabile creata non viene passara al child schell. per faglielo conoscere si usa il biltin export

in un processo sappiamo che possiamo vedere le varaiibli di ambiente anche in runtime ma non nello stesso punto in cui si trovano. guardndo fig 7.5 tutti i valori, ogniuno separato da un null $\backslash 0$ stara in un array di puntatori a quisti valori, aray che termina con un null e tramire quest puntator iarivaimo agle varaiibli di ambiente.

i porgrammi posson oallora creare un environment pointer tramite un handle (punttaore a puntatori) al nostro array detto environment list. 

\begin{lstlisting}
extern char **environ;
\end{lstlisting}

in alternativa si pu`o usare o un valore *envp[] nei parametri dei main, oppure si usa la funzine 

\begin{lstlisting}
char *getenv(const vhar *name);
\end{lstlisting}

passando direttamente il nome della variaible di ambiente restituendo un puntatore al suo contenuto

abbiamo che la malloc usa zone globali allora non è rientrant (vediamo poi cosa vuol dire)

(VEDERE memory\_dump.c) che vede e sampa tutte le variabili di mabinete, aspetta un p`o e poi lo ri fa stampandone anche di nuove e modificandone una esistente e poi va a vedere rispetto ai punttori precendenti c se sno ancora buoni  o se stanno da un altra parte 

vedere a mush simpler way to print addresses used in process memory per stampare semplicemtne gli indirizzi delle variabili di ambiente