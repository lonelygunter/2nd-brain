\newpage
\section{Introduzione}

% System call
\subsection{System call}

Sono \hl{uguali alle funzioni di libreria dal punto di vista sintattico, pero' cambia il modo di compilarle}. Notare che non possono essere usati i nomi delle SC per delle function call.

Per poi poter "raccontare" tra umani le sequenze di bit che vengono mandate ai processori si usa \hl{assembly}.

Sono effettivamente delle chiamate a funzioni ma poi dal codice assembly puoi capire che è una system call dato che ha dei meccanismi specifici.

Alcuni esempi di chiamate e registri:
\begin{itemize}
	\item \textbf{eax} : registro dove metti il \textbf{numero della sc}
	\item \textbf{int 0x80}: \textbf{avvisa il kernel} che serve chiamare una sc
	\item \textbf{exit()}: chiudere un processo
	\item \textbf{write()}:
\end{itemize}

\begin{lstlisting}
mov edx,4    ; lunghezza messaggio
mov ecx,msg  ; puntatore al messaggio
mov ebx,1    ; file descriptor
mov eax,4    ; numero della sc
int 0x80	
\end{lstlisting}

dove nel \hl{file descriptor} indichi a quale file devi mandare l'output. Questo viene usato dato che così non deve cercare il path ogni volta ma lo mantiene aperto \hl{riferendosi ad esso tramite un numero, cioe' il piu' piccolo disponibile}.


% Programma Make
\subsection{Programma Make}

Quando viene avviato verifica la presenza di un file chiamato "Makefile", oppure si usa 'make -f'. In questo file ci sono le \hl{regole di cosa fare per automatizzare delle azioni per un numero n di file}. Se, durante la compilazione di massa, \hl{una di queste da un errore il programma make si interrompe}, per evitare ciò si usa '-i' (ignore).

Il Makefile andrà ad aggiornare una libreria andando a guardare se una delle 3 date di ultima modifica si sono aggiornate.


Andiamo a guardare \hl{cosa contiene Makefile}:

\begin{lstlisting}
DIRS = lib intro sockets advio daemons datafiles db environ \
	fileio filedir ipc1 ipc2 proc pty relation signals standards \
	stdio termios threadctl threads printer exercises

all:
	for i in $(DIRS); do \
		(cd $$i && echo "making $$i" && $(MAKE) ) || exit 1; \
	done

clean:
	for i in $(DIRS); do \
		(cd $$i && echo "cleaning $$i" && $(MAKE) clean) || exit 1; \
	done
\end{lstlisting}

dove:
\begin{itemize}
	\item \hl{DIRS}: lo si associa alle \textbf{stringhe singole} che gli sono state associate
	\item \hl{all}: nel ciclo for:
		\begin{itemize}
			\item manda un comando in subshell
			\item \$\$i: riferimento alla variabile "i" del for + simbolo escape per il Makefile
			\item \$(MAKE): macro predefinita per i Makefile
		\end{itemize}
 
\end{itemize}

La struttura è:

\begin{lstlisting}
target: prerequisiti
	rule
\end{lstlisting}

dove:

\begin{itemize}
	\item \hl{target}: è l\textbf{a cosa che si vuole fare}, se essendo il primo target, sarà anche quello di default
	\item \hl{prerequisiti}: \textbf{file e/o target} a loro volta
	\item \hl{rule}: indica \textbf{cosa puo' fare il target}
\end{itemize}

Può capitare che prima di eseguire il Makefile ci sia uno \hl{script "configure"}. 

In molti casi si ha un \hl{target "clean"} che permette di pulire i file .o che sono inutili dopo la compilazione, o comunque qualsiasi tipo di file gli si voglia far eliminare. Questo tipo di target che non rappresentano un file, sono detti \hl{"phony"} perchè fasulli, dato che non sono file ma sole parole

\begin{lstlisting}
file: file.o lib.o

clean:
	rm file.o
\end{lstlisting}

Abbiamo delle \hl{variabili automatiche} per rendere il lavoro più facile:

\begin{itemize}
	\item \textbf{\$@}: per riferirsi il target
	\item \textbf{\$?}: tutti i prerequisiti più recenti del target
	\item \textbf{\$\^}: tutti i prerequisiti del target
	\item \url{https://www.gnu.org/software/make/manual/make.html#Automatic-Variables}
\end{itemize}


Un altro esempio di Makefile è:

\begin{lstlisting}
ROOT=..
PLATFORM=$(shell $(ROOT)/systype.sh)
include $(ROOT)/Make.defines.$(PLATFORM)

PROGS =	getcputc hello ls1 mycat shell1 shell2 testerror uidgid

all:	$(PROGS)

%:	%.c $(LIBAPUE)
	$(CC) $(CFLAGS) $@.c -o $@ $(LDFLAGS) $(LDLIBS)

clean:
	rm -f $(PROGS) $(TEMPFILES) *.o

include $(ROOT)/Make.libapue.inc
\end{lstlisting}

dove:

\begin{itemize}
	\item ROOT: cwd
	\item PLATFORM: assumera in valore del OS: macos/linux
	\item include: include un file
	\item PROGS: elenco dei programmi da usare
	\item \%: target con nome variabile, indica un file
	\item \%.c: target con nome variabile ma estensione .c
	\item \$(CC): indica il compilatore dove cc è un link simbolico a clang
	\item \$(CFLAGS) indica una macro predefinita dei default vuota che si può usare all'occorrenza
	\item all: target che prende in carico tutti i programmi che se saranno di tipo .c saranno presi in carico dal target successivo
\end{itemize}


\hl{Per la compilazione dei file}, qualsiasi sia il linguaggio, \hl{make sapra' come compilarlo} grazie a tutte le definizioni di default presenti in:

\begin{lstlisting}
$ make -p
\end{lstlisting}

notare che \textbf{si puo' mettere un comando custom nelle rule} del target

Nell'eventualità di \hl{voler aggiornare un solo file della libreria senza far aggiornare il resto ci bastera' usare uno script} che compila quel file passato a linea di comando.

\begin{lstlisting}
$ gcc -ansi -I../include -Wall -DMACOS -D_DARWIN_C_SOURCE  ${1}.c -o ${1}  -L../lib -lapue
\end{lstlisting}



% Direttive di preprocessore
\subsection{Direttive di preprocessore}
Sono delle \hl{indicazioni date a gcc prima di iniziare la compilazione}.

Iniziano tutte con '\#':
\begin{itemize}
	\item \textbf{\#include}: serve ad \textbf{includere delle librerie} di sistema ($<$lib.h$>$) oppure di librerie fatte da noi e non in directory standard ("lib.h")
	\item \textbf{\#define}:
		\begin{itemize}
			\item permette di \textbf{creare delle "macro"}, che vanno a sostituire una stringa con un'altra (es: \#define BUFLEN), può capitare che debbano essere definite delle macro prima che si compili il programma, in questi casi si usa scrivere es: '-DMACOS'
			\item permette di \textbf{creare delle "function like macro"} (es: \#define ABSOLUTE\_VALUE(x) (((x$<$0)?-(x):(x))  
		\end{itemize}
		
	\item \textbf{\#ifdef, \#ifndef, \#endif}: usata per far accadere qualcosa nel caso un macro sia stata definita
\begin{lstlisting}
#ifdef VAR
print("hello");
#endif
\end{lstlisting}

\end{itemize} 

\hl{Per evitare che piu' file includano lo stesso si usano degli \#ifndef} in tutto il codice, in modo da evitare doppie definizioni.


% Librerie
\subsection{Librerie}

Durante la fase di compilazione creiamo dei file oggetto (.o) per ogni file in cui è scritta la descrizione delle funzioni di libreria (.c)

\begin{lstlisting}
$ gcc -c bill.c
\end{lstlisting}

Si andrà poi a creare il \hl{prototipo della funzione (.h)}.

In fine \hl{tramite il linker si andranno ad unire tutti i file per crearne uno unico} con tutte le definizioni delle funzioni incluse nelle librerie, di sistema e non, importate. Si vanno quindi a \hl{sciogliere tutti i riferimenti incrociati}.

\begin{lstlisting}
$ gcc -o program program.o bill.o
\end{lstlisting}

Per quanto riguarda le \hl{funzioni di sistema} NON abbiamo il file sorgente ma abbiamo direttamente l'eseguibile. In compenso abbiamo un \hl{file di libreria}, cioè un insieme di file oggetto linkati in un unico file, dove c'è il codice oggetto di tutte le funzioni.

Abbiamo \textbf{2 tipi di librerie}:
\begin{itemize}
	\item \hl{statiche}: è una \textbf{collezione di file oggetto} che hanno il codice compilato delle funzioni e che verranno \textbf{linkati al momento della compilazione}. Il programma che si crea sarà possibile essere eseguito solo sullo stesso OS.
		
		Il \textbf{problema si ha nell'aggiornamento delle librerie al momento della scoperta di un bug}. Una volta coretto servirà ricevere la versione corretta per poter aggiornare il programma.
		
	\item \hl{dynamic}: ricordano il concetto di plug-in, quindi \textbf{viene invocato a runtime e caricato in memoria} (es: aggiornamenti dei OS). \textbf{L'eseguibile non viene toccato la correzione avviene solo nella libreria}.
	
		Il requisito maggiore è che chi si passa il codice debba avere lo stesso OS dell'altro utente. Notare che \textbf{non cambia il prototipo} dato che sennò bisognerà ricompilare l'intero programma.
\end{itemize}


In generale le \hl{librerie statiche sono molto pericolose} infatti alcuni OS le aboliscono \hl{per le questioni di sistema}. Su linux si ha come libreria statica 'lib.c' che è la libreria con le funzioni più usate in c. Per macos è stata abolita.

Per compilare con la versione dinamica non servono opzioni, per la statica si usa:

\begin{lstlisting}
$ gcc -static
\end{lstlisting}


% Creazione librerie
\subsection{Creazione librerie}

Per costruire una \hl{libreria statica per MacOS}:

\begin{enumerate}
	\item costruiamo il \textbf{file oggetto}:

\begin{lstlisting}
$ gcc -c libprova.c
\end{lstlisting}
		
	\item costruiamo la \textbf{libreria} (con ar=archive, c=create se lib.a non esiste):

\begin{lstlisting}
$ ar rcs libprova.a libprova.o
\end{lstlisting}
			
	\item costruire il \textbf{codice} che usa la libreria (con -Wall=verbose warning, -g=debugging, -c=create del file):

\begin{lstlisting}
$ gcc -Wall -g -c useprova.c
\end{lstlisting}
	
	\item \textbf{linker} per risolve le chiamate incrociate (con -L.=dove prendere la libreira, -l[nomelib]=usare la libreria):

\begin{lstlisting}
$ gcc -g -o useprova useprova.o -L. -lprova 
\end{lstlisting}

\end{enumerate}

Per capire che librerie usa il codice si usa:

\begin{lstlisting}
$ otool -L [nomecodice]
\end{lstlisting}


Per costruire una \hl{libreria statica per Linux}:

\begin{enumerate}
	\item costruiamo il \textbf{file oggetto}:

\begin{lstlisting}
$ gcc -fPIC -Wall -g -c libprova.c
\end{lstlisting}

	\item costruiamo la \textbf{libreria} (con 0.0=versione della libreira):

\begin{lstlisting}
$ gcc -g -shared -Wl,-soname,libprova.so.0 -o libprova.so.0.0 libprova.o -lc 
\end{lstlisting}

	\item costruire il \textbf{link simbolico per aggiornare le librerie} senza aggiornare gli eseguibili e senza cambiare il nome del programma:

\begin{lstlisting}
$ ln -sf libprova.so.0.0 libprova.so.0 
\end{lstlisting}

	\item \textbf{linker} per risolve le chiamate:

\begin{lstlisting}
$ ln -sf libprova.so.0 libprova.so
\end{lstlisting}

\end{enumerate}


Per capire che librerie usa il codice si usa:

\begin{lstlisting}
$ ldd [nomevodice]
\end{lstlisting}


% Aggiornamento librerie
\subsection{Aggiornamento librerie}

Su \textbf{Linux} il sistema \hl{andra' a prendere direttamente una libreria dinamica}, per evitare ciò e far trovare la nostra, basterà \hl{impostare una variabile di ambiente}:

\begin{lstlisting}
LD_LIBRARY_PATH=`pwd` ldd useprova
\end{lstlisting}

Tipicamente la libreria viene distribuita nelle directory di sistema andandola ad "installare".


Su \textbf{MacOS} la libreria dinamica è un \hl{.dylib}:

\begin{lstlisting}
$ gcc -dynamiclib libprova.c -o libprova.dylib
\end{lstlisting}

Quindi eseguendo il programma \hl{trovera' la libreria controllando nella directory corrente} e quindi non serve creare la variabile di ambiente come su Linux.

i file di intestazione del mac come stdio.h per cercarla uso:

\begin{lstlisting}
$ find /Applications/Xcode.app/ -name stdio.h 2>/dev/null
\end{lstlisting}


