
\section{Introduzione - 23/27.09.22}

% System call
\subsection{System call}

Sono \hl{uguali alle funzioni di libreria dal punto di vista sintattico, pero' cambia il modo di compilarle}. Notare che non possono essere usati i nomi delle SC per delle function call.

Per poi poter "raccontare" tra umani le sequenze di bit che vengono mandate ai processori si usa \hl{assembly}.

Sono effettivamente delle chiamate a funzioni ma poi dal codice assembly puoi capire che è una system call dato che ha dei meccanismi specifici.

Alcuni esempi di chiamate e registri:
\begin{itemize}
	\item \textbf{eax} : registro dove metti il \textbf{numero della sc}
	\item \textbf{int 0x80}: \textbf{avvisa il kernel} che serve chiamare una sc
	\item \textbf{exit()}: chiudere un processo
	\item \textbf{write()}:
\end{itemize}

\begin{lstlisting}
mov edx,4    ; lunghezza messaggio
mov ecx,msg  ; puntatore al messaggio
mov ebx,1    ; file descriptor
mov eax,4    ; numero della sc
int 0x80	
\end{lstlisting}

dove nel \hl{file descriptor} indichi a quale file devi mandare l'output. Questo viene usato dato che così non deve cercare il path ogni volta ma lo mantiene aperto riferendosi ad esso tramite il numero.


% Programma Make
\subsection{Programma Make - 28.09.22}

Quando viene avviato verifica la presenza di un file chiamato "Makefile", oppure si usa 'make -f'. In questo file ci sono le \hl{regole di cosa fare per automatizzare delle azioni per un numero n di file}. Se, durante la compilazione di massa, \hl{una di queste da un errore il programma make si interrompe}, per evitare ciò si usa '-i' (ignore).


Andiamo a guardare \hl{cosa contiene Makefile}:

\begin{lstlisting}
DIRS = lib intro sockets advio daemons datafiles db environ \
	fileio filedir ipc1 ipc2 proc pty relation signals standards \
	stdio termios threadctl threads printer exercises

all:
	for i in $(DIRS); do \
		(cd $$i && echo "making $$i" && $(MAKE) ) || exit 1; \
	done

clean:
	for i in $(DIRS); do \
		(cd $$i && echo "cleaning $$i" && $(MAKE) clean) || exit 1; \
	done
\end{lstlisting}

dove \hl{all e' detto target}, cioè la cosa che si vuole fare, eseguiremo allora un "make all". \textbf{Essendo il primo target, sarà anche quello di default}.

Possono essere presenti dei \hl{prerequisiti}, dopo i ":", che possono essere a loro volta dei target.

Obbligatoriamente avremo, dopo i prerequisiti, la \hl{riga delle regole} che \hl{indica cosa puo' fare il target}.