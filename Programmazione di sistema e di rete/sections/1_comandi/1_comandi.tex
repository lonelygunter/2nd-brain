\section{Comandi utili}

% find
\subsection{find: trovare tutti i file eseguibili}
	
\begin{lstlisting}
$ find . -type f -perm -0100
./standards/makeopt.awk
./standards/makeconf.awk
./proc/awkexample
./systype.sh
./advio/fixup.awk
\end{lstlisting}


\subsection{find: trovare file di intestazione del mac come stdio.h}

\begin{lstlisting}
find /Applications/Xcode.app/ -name stdio.h 2>/dev/null
\end{lstlisting}


% ldd
\subsection{lld: per capire che librerie usa il codice}

\begin{lstlisting}
ldd [nomevodice]
\end{lstlisting}


% gcc
\subsection{gcc: per vedere tutta la gerarchia di file in una libreria}

\begin{lstlisting}
gcc -H lib.a
\end{lstlisting}


\subsection{gcc: per vedere il codice con tutti i file importati}

\begin{lstlisting}
gcc -E file.c
\end{lstlisting}


\subsection{gcc -g: debugging debole}

\begin{lstlisting}
gcc -g -ansi -I../include -Wall -DMACOS -D_DARWIN_C_SOURCE  ls1.c -o ls1  -L../lib -lapue
\end{lstlisting}


\subsection{gcc -ggbd: debugging forte}

\begin{lstlisting}
gcc -ggbd -ansi -I../include -Wall -DMACOS -D_DARWIN_C_SOURCE  ls1.c -o ls1  -L../lib -lapue
\end{lstlisting}


% xattr
\subsection{xattr: usato per i file che entrano in quarantena su MacOS}

\begin{lstlisting}
xattr -d (delete) com.apple.quarantine [path sh]
\end{lstlisting}





\section{Variabili di sistema definite in .bashrc}

% INC
\subsection{INC}

\begin{lstlisting}
INC="/Applications/Xcode.app/Contents/Developer/Platforms/MacOSX.platform/Developer/SDKs/MacOSX.sdk/usr/include/"
\end{lstlisting}