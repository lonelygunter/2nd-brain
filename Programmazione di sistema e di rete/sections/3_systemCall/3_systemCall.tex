\newpage
\section{System call}

% Funzioni e system call
\subsection{Funzioni e system call}

Se prendiamo un funzionamento più semplice del comando "ls" potrebbe essere:

\begin{lstlisting}
#include "apue.h"
#include <dirent.h>

int
main(int argc, char *argv[])
{
	DIR				*dp;
	struct dirent	*dirp;

	if (argc != 2)
		err_quit("usage: ls1 directory_name");

	if ((dp = opendir(argv[1])) == NULL)
		err_sys("can't open %s", argv[1]);
	while ((dirp = readdir(dp)) != NULL)
		printf("%s\n", dirp->d_name);

	closedir(dp);
	exit(0);
}
\end{lstlisting}


dove abbiamo che:

\begin{itemize}
	\item \hl{DIR}: struttura dati
	\item \hl{struct dirent}: \textbf{tipo struttura} che contiene al suo interno diversi tipi di variabili. 
	
		Per capire se è una funzione di sistema lanciamo:
		
\begin{lstlisting}
$ grep -rw "struct dirent" $INC

/Applications/Xcode.app/Contents/Developer/Platforms/MacOSX.platform/Developer/SDKs/MacOSX.sdk/usr/include//sys/dirent.h:struct dirent {
\end{lstlisting}
		
\begin{lstlisting}
#ifndef _SYS_DIRENT_H
#define _SYS_DIRENT_H

#include <sys/_types.h>
#include <sys/cdefs.h>

#include <sys/_types/_ino_t.h>


#define __DARWIN_MAXNAMLEN      255

#pragma pack(4)

#if !__DARWIN_64_BIT_INO_T
struct dirent {
	ino_t d_ino;                    /* file number of entry */
	__uint16_t d_reclen;            /* length of this record */
	__uint8_t  d_type;              /* file type, see below */
	__uint8_t  d_namlen;            /* length of string in d_name */
	char d_name[__DARWIN_MAXNAMLEN + 1];    /* name must be no longer than this */
};
#endif /* !__DARWIN_64_BIT_INO_T */

#pragma pack()

#define __DARWIN_MAXPATHLEN     1024

#define __DARWIN_STRUCT_DIRENTRY { \
	__uint64_t  d_ino;      /* file number of entry */ \
	__uint64_t  d_seekoff;  /* seek offset (optional, used by servers) */ \
	__uint16_t  d_reclen;   /* length of this record */ \
	__uint16_t  d_namlen;   /* length of string in d_name */ \
	__uint8_t   d_type;     /* file type, see below */ \
	char      d_name[__DARWIN_MAXPATHLEN]; /* entry name (up to MAXPATHLEN bytes) */ \
}

#if __DARWIN_64_BIT_INO_T
struct dirent __DARWIN_STRUCT_DIRENTRY;
#endif /* __DARWIN_64_BIT_INO_T */



#if !defined(_POSIX_C_SOURCE) || defined(_DARWIN_C_SOURCE)
#define d_fileno        d_ino           /* backward compatibility */
#define MAXNAMLEN       __DARWIN_MAXNAMLEN
/*
 * File types
 */
#define DT_UNKNOWN       0
#define DT_FIFO          1
#define DT_CHR           2
#define DT_DIR           4
#define DT_BLK           6
#define DT_REG           8
#define DT_LNK          10
#define DT_SOCK         12
#define DT_WHT          14

/*
 * Convert between stat structure types and directory types.
 */
#define IFTODT(mode)    (((mode) & 0170000) >> 12)
#define DTTOIF(dirtype) ((dirtype) << 12)
#endif


#endif /* _SYS_DIRENT_H  */
\end{lstlisting}

		dove vediamo che se la variabile "\_\_DARWIN\_64\_BIT\_INO\_T" è stata definita avremo che la struttura di struct dirent è:
		
\begin{lstlisting}
#define __DARWIN_STRUCT_DIRENTRY { \
	__uint64_t  d_ino;      /* file number of entry */ \
	__uint64_t  d_seekoff;  /* seek offset (optional, used by servers) */ \
	__uint16_t  d_reclen;   /* length of this record */ \
	__uint16_t  d_namlen;   /* length of string in d_name */ \
	__uint8_t   d_type;     /* file type, see below */ \
	char      d_name[__DARWIN_MAXPATHLEN]; /* entry name (up to MAXPATHLEN bytes) */ \
}

#if __DARWIN_64_BIT_INO_T
struct dirent __DARWIN_STRUCT_DIRENTRY;
#endif /* __DARWIN_64_BIT_INO_T */
\end{lstlisting}

		
	\item \hl{if}: esegue un controllo sugli args. Notiamo che "err\_quit" non è una funzione di sistema da:
	
\begin{lstlisting}
$ grep -rw "err_quit" $INC
\end{lstlisting}

		infatti non restituisce nulla. Deve allora essere una funzione di libreria create da noi quindi non presente nella directory standard.
		
		La funzione andrà a dare un messaggio di errore e poi esce dal programma.
		
		
	\item \hl{opendir}: serve ad aprire una directory andandola a caricare nella RAM. 
	
	\item \hl{while}: leggiamo la directory e la inseriamo nella struttura che poi sarà richiamata tramite:
	
\begin{lstlisting}
dirp->d_name
\end{lstlisting}

		dove "d\_name" è il nome dello slot in cui è contenuto il nome del file.
		
		
	\item \hl{exit}: restituisce l'exit code del programma
	
\end{itemize}


% Capire se una funzione è una system call
\subsection{Capire se una funzione è una system call}

Andiamo a \hl{vedere se e' una funzione o una system call tramite "man"}, lo si capisce tramite la dicitura in alto alla pagina del manuale:

	\begin{itemize}
		\item \textbf{Library Functions} Manual
		\item \textbf{System Calls} Manual
	\end{itemize}

Abbiamo anche \hl{esempi piu' particolari}, come fork, dove è indicata come system call ma in realtà le richiama ma in prima persona.


Potremo trovare i simboli di una libreria tramite:

\begin{lstlisting}
$ nm lib.a
\end{lstlisting}

che ci fa vedere, per ogni file oggetto, i simboli associati per ogni funzione.

Le system call le troveremo in "\$INC/sys/syscall.h"


% Numeri dei file descriptor
\subsection{Numeri dei file descriptor}

Prendiamo un esempio semplificato del comando "cat":

\begin{lstlisting}
#include "apue.h"

#define	BUFFSIZE	4096

int
main(void)
{
	int		n;
	char	buf[BUFFSIZE];

	while ((n = read(STDIN_FILENO, buf, BUFFSIZE)) > 0)
		if (write(STDOUT_FILENO, buf, n) != n)
			err_sys("write error");

	if (n < 0)
		err_sys("read error");

	exit(0);
}
\end{lstlisting}

\hl{ogni processo ha 3 file descriptor usati 0, 1, 2}.

\begin{itemize}
	\item \hl{BUFFSIZE}: macro di preprocessore
	\item \hl{read}: \textbf{system call} con parametri: 
	
		\begin{itemize}
			\item \textbf{STDIN\_FILENO}: file descriptor per dire da quale "numero di deve leggere" si vuole leggere. Cioè per leggere dal file indicato nello standard input
			\item \textbf{buf}: indirizzo dell'\textbf{inizio dell'array}
			\item \textbf{BUFFSIZE}: quando deve leggere
		\end{itemize}
		
		\hl{Restituisce il numero di char che ha letto}, dato che potrebbe leggere meno byte di quelli richiesti nel caso in cui il file ne contenga di meno. Ad ogni sua iterazione \hl{si ricorda la "posizione nel file"} che gli permette di non leggere sempre i primi n byte ma di rincominciare da dove ha lasciato.
	
	\item \hl{write}: richiede gli stessi valori di read tranne per \textbf{STDOUT\_FILENO} e \textbf{ritorna il numero byte effettivamente letti}
	\item 
\end{itemize}


per capire \hl{quanto vale STDIN\_FILENO}:


\begin{lstlisting}
$ grep -rw "STDIN_FILENO" $INC

/Applications/Xcode.app/Contents/Developer/Platforms/MacOSX.platform/Developer/SDKs/MacOSX.sdk/usr/include//unistd.h:#define	 STDIN_FILENO	0	/* standard input file descriptor */
/Applications/Xcode.app/Contents/Developer/Platforms/MacOSX.platform/Developer/SDKs/MacOSX.sdk/usr/include//asl.h: * asl_log_descriptor(c, m, ASL_LEVEL_NOTICE, STDIN_FILENO, ASL_LOG_DESCRIPTOR_READ);
\end{lstlisting}


Sappiamo che un processo per eseguire un programma, esegue prima una \hl{fork} e poi con \hl{exec} esegue il programma. Prima di eseguire la fork il \hl{child chiude il file 1} e quando fa una \hl{open}, la system call prenderà il file nel quale reindirizzare lo STDOUT e restituirà il numero 1.

Su questo sistema si base UINX infatti avviene anche con le pipe "|". Permette di creare programmi complessi unendo tanti piccoli programmi specializzati in un'unica funzione.

È molto importante capire che \hl{ i child ereditano i file descriptor dei parent} quindi non è necessario che il programma corrente faccia una open dei file descriptor.


% Meccanismi dei file
\subsection{Meccanismi dei file}

Un file è in insieme di meccanismi: 

\begin{itemize}
	\item \hl{apri}
	\item \hl{leggi}
	\item \hl{scrivi}
	\item \hl{chiudi}
\end{itemize}


Questi meccanismi sono applicabili a file, cartelle, stampanti ecc..., solo che per ogni "tipo" \hl{i 4 meccanismi si adeguano} a ciò che il caso particolare deve fare.


% Unbuffered I/O
\subsection{Unbuffered I/O}

Le system call rappresentano una barriera tra kernel e programmi, dove avremo rispettivamente \hl{due diverse modalita' di utilizzo}:

\begin{itemize}
	\item \textbf{kernel mode}: ha tutti i privilegi
	\item \textbf{user mode}: non può accedere a tutte le celle di memoria
\end{itemize}


Per \hl{ottimizzare la scrittura sulla memoria da parte del kernel si utilizza la libreria STDIOLIB} che incrementa le prestazioni dato che gestisce il passaggio di pacchetti con il kernel in modo da inviare dei pacchetti consistenti ogni tot e non piccoli pacchetti soni secondo. Per fare ciò usa un \hl{buffered i/o} che, una volta riempiti dei buffer, gli manda al kernel.


\begin{figure}[H]
\centering
\includegraphics[scale=0.4]{unbuffio.jpeg}
\caption{Schema unbuffered I/O} 
\label{unbuffio}
\end{figure}


% Fork & Exec
\subsection{Fork \& Exec}

Prendiamo il codice di shell1.c che crea uno schell dal quale poter eseguire programmi:

\begin{lstlisting}
#include "apue.h"
#include <sys/wait.h>

int
main(void)
{
	char	buf[MAXLINE];	/* from apue.h */
	pid_t	pid;
	int		status;

	printf("%% ");	/* print prompt (printf requires %% to print %) */
	while (fgets(buf, MAXLINE, stdin) != NULL) {
		if (buf[strlen(buf) - 1] == '\n')
			buf[strlen(buf) - 1] = 0; /* replace newline with null */

		if ((pid = fork()) < 0) {
			err_sys("fork error");
		} else if (pid == 0) {		/* child */
			execlp(buf, buf, (char *)0);
			err_ret("couldn't execute: %s", buf);
			exit(127);
		}

		/* parent */
		if ((pid = waitpid(pid, &status, 0)) < 0)
			err_sys("waitpid error");
		printf("%% ");
	}
	exit(0);
}
\end{lstlisting}

avremo allora:

\begin{itemize}
	\item \hl{fgets}: funzione dello \textbf{stdoutput che legge la stringa} che dai prima di dare invio, ha come argomenti:
	
		\begin{itemize}
			\item \textbf{buf}: buffer nel quale mettere la stringa
		
			\item \textbf{MAXLINE}: proviene da una nostra libreria

\begin{lstlisting}
$ grep -rw "MAXLINE" include/
Binary file include//apue.h.gch matches
include//apue.h:#define	MAXLINE	4096			/* max line length */
\end{lstlisting}

			\item \textbf{stdin}: presente in stdiolib ed è una \textbf{struttura file che definisce uno standard input} tramite un puntatore ad un "file"
		\end{itemize}

	\item \hl{if 1}: permette di avere un null dove prima avevamo \textbackslash n
	
	\item \hl{if 2}: abbiamo una fork() che dopo che \hl{viene invocata ritorna 2 volte}, questo perché andrà a creare 2 bash identici con memorie uguali nei contenuti ma indipendenti, l'unico cambiamento è il pid. fork() andrà quindi a restituire 0 nel child e il pid del child al parent tramite getppid().

			Se pid $<$ 0 vorrà dire che la fork e fallita.
			Se pid = 0 vorrà dire che siamo nel child.
	
			Il che è molto importante dato che \hl{il codice verra' eseguito sia dal child che dal parent}, e sarà contenuto nella memora virtuale che hanno i programmi grande $2^{32}$ o $2^{64}$ in base all'OS.
	
			Appena viene eseguita l'\hl{exec, lo spazio di memoria viene azzerato} ma a discrezione del programma, vengono salvate alcune variabili di ambiente.


		\hl{execlp}: serve a far \textbf{eseguire un codice} (buf) del quale abbiamo il sorgente e l'eseguibile
	

		\hl{if 3}: serve a far andare aventi il parent.

		\hl{waitpid}: aspetta il child nel caso impieghi troppo tempo ad eseguire la sua azione, \textbf{tenendo appeso il prompt}. Con argomenti:

			\begin{itemize}
				\item \textbf{pid}: pid del child
				\item \textbf{\&status}: reindirizza l'exit code del child, quando finisce, nella variabile status
			\end{itemize}
			
\end{itemize}


\begin{figure}[H]
\centering
\includegraphics[scale=0.7]{fork.jpeg}
\caption{Esecuzione di fork ed exec} 
\label{fork}
\end{figure}


% Thread
\subsection{Thread}

Sono dei \hl{processi con lo stesso spazio di memoria del parent}. Per evitare che ogniuno scriva dove vuole, \hl{avviene una sincronizzazione tra le thread}. Questo metodo viene usato nelle macchine unicore per poter svolgere più operazione "contemporaneamente".

Tutto questo è \hl{orchestrato dal kernel} che gestisce il \hl{time shearing}.


% Gestione degli errori
\subsection{Gestione degli errori}

Per convenzione una funzione ritorna 0 se è andato tutto bene. Ci sono delle eccezioni, come la read, che ritorna il numero di byte letti.

Ogni valore possibile ritornato è specificato nel manuale:

\begin{lstlisting}
$ man 2 intro
...
1 EPERM Operation not permitted. An attempt was made to perform an operation limited to processes with appropriate privileges or to the owner of a file or other resources.

2 ENOENT No such file or directory. A component of a specified pathname did not exist, or the pathname was an empty string.
...
\end{lstlisting}


Per le \hl{system call}, quando avviene un errore, si \hl{avvalora la variabile "errno"} che può essere consultata in un programma con 

\begin{lstlisting}
extern int errno;
\end{lstlisting}


Con "errno" bisogna tenere in conto che:

\begin{enumerate}
	\item \hl{non viene svuotata quando passiamo l'errore}. Quindi per sapere quando è stato dato un errore bisogna andare a consultarla quando la system call viene invocata
	\item \hl{vale 0} se non usata
\end{enumerate}


Per la gestione degli errori useremo:

\begin{itemize}
	\item \hl{strerror}: restituisce la \textbf{stringa del valore} di errno
	\item \hl{perror}: legge errno e stampa un messaggio a piacere
\end{itemize}


\begin{lstlisting}
#include "apue.h"
#include <errno.h>

int
main(int argc, char *argv[])
{
	fprintf(stderr, "EACCES: %s\n", strerror(EACCES));
	errno = ENOENT;
	perror(argv[0]);
	exit(0);
}
\end{lstlisting}


dove:

\begin{itemize}
	\item \hl{fprinf}: stampa un errore allo standard specificato
\end{itemize}



% 