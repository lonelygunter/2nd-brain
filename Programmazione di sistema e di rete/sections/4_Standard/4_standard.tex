\newpage
\section{Gli standard}

% Storia e basi
\subsection{Storia e basi}

La \hl{standardizzazione di UNIX e' iniziata nel 1988} facendo affidamento ad alcuni \hl{standard di C} dato che fa usi di interfaccie e prototipi.

In definitiva abbiamo gli standard:

\begin{itemize}
	\item Posix.1-2001 / SUSv3: (\url{http://pubs.opengroup.org/onlinepubs/009604599/})
	\item Posix.1-2008 / SUSv4:  più usato in ambiti di automazioni aziendali, infatti sono specializzate sullo scambio di informazione in segnali realtime. Per questo la sua certificazione non è stata presa da nessuno se non fa un IBM. (\url{http://pubs.opengroup.org/onlinepubs/9699919799/})
\end{itemize}

(\hl{PS: le versioni sono back compatibili} quindi se settiamo -D\_XOPEN\_SOURCE=700 non precludiamo la SUSv3)

Nonstante gli standard \hl{ogni OS fa delle sue modifiche su alcune cose esterne alle SUS}.

Per verificare il tipo di standard su un applicativo (\_XOPEN\_SOURCE) o un sistema (\_XOPEN\_VERSION), si fa affidamento alle "\hl{feature test macros}" consultabili dai \textbf{codici di intestazione .h}. Per esempio \_XOPEN\_SOURCE impostata a 600 o 700 indica SUSv3 o SUSv4.

\begin{lstlisting}
-D_XOPEN_SOURCE=600
\end{lstlisting}

in questo modo potremo allora andare a compilare tutti i programmi conformi su qualsiasi OS.


% Limiti
\subsection{Limiti}

abbimao dei limiti di compilazione che possono essere visti nei file di inestazione

runtime limit: si vedono tramite la ufnzione sysconf (es: lunghezza massima del nome dei file che dipende dal filesystem può capirlo tramite pathconf su un file qualunque di quel filesystem)

sysconf:
