\documentclass[journal]{IEEEtran}
\IEEEoverridecommandlockouts

%%%%%%%%%%%%%%%%%%%%%%%%%%%%%%%%%%%%%%
%%%%%%%% PACCHETTI PRINCIPALI %%%%%%%%
%%%%%%%%%%%%%%%%%%%%%%%%%%%%%%%%%%%%%%
\usepackage{fancyhdr}
\usepackage{graphicx}
\usepackage[italian]{babel}
\usepackage[utf8]{inputenc}
\usepackage{color}
\usepackage{hyperref}
\usepackage{wrapfig}
\usepackage{array}
\usepackage{multirow}
\usepackage{adjustbox}
\usepackage{nccmath}
\usepackage{subfigure}
\usepackage{amsfonts,latexsym}
\usepackage{enumerate}
\usepackage{booktabs}
\usepackage{float}
\usepackage{threeparttable}
\usepackage{array,colortbl}
\usepackage{ifpdf}
\usepackage{rotating}
\usepackage{cite}
\usepackage{stfloats}
\usepackage{url}
\usepackage{listings}
\usepackage{soul}

%%%%%%%%%%%%%%%%%%%%%%%%%%%%%%%%%%%%
%%% CREA E SCRIVI ALCUNI COMANDI %%%
%%%%%%%%%%%%%%%%%%%%%%%%%%%%%%%%%%%%
\newcolumntype{P}[1]{$>${\centering\arraybackslash}p{#1}}  %% Viene creato un nuovo tipo di colonna denominata P.

% correggere la sillabazione errata qui
\hyphenation{op-tical net-works semi-conduc-tor} %% Con questo comando si specifica come separare correttamente le sillabe nel caso in cui una parola si trovi in due diverse righe di testo

\graphicspath{ {img/} }  %%Percorso dove si trovano le immagini, se è vuoto indica che le immagini sono all'interno della stessa cartella che contiene il file .tex


%%%%%%%%%%%%%%%%%%%%%%%%%%%%%%%%%%%%%%%%%%%%%%%%
%%% INTESTAZIONE DELLE PAGINE TIPO UNICAFAM %%%%
%%%%%%%%%%%%%%%%%%%%%%%%%%%%%%%%%%%%%%%%%%%%%%%%
\newcommand{\MYhead}{\smash{\scriptsize
\hfil\parbox[t][\height][t]{\textwidth}{\centering
\begin{picture}(0,0) \put(-30,-13){\includegraphics[width=30mm]{logoUnisalento.jpg}} \end{picture} \hspace{6.4cm}
INGEGNERIA INFORMATICA \\
\hspace{5.2cm} DIPARTIMENTO DI INGEGNERIA DELL'INNOVAZIONE \hspace{3cm} \\
\underline{\hspace{ \textwidth}}}\hfil\hbox{}}}
\makeatletter

% normal pages
\def\ps@headings{%
\def\@oddhead{\MYhead}%
\def\@evenhead{\MYhead}}%

% title page
\def\ps@IEEEtitlepagestyle{%
\def\@oddhead{\MYhead}%
\def\@evenhead{\MYhead}}%
\makeatother

% make changes take effect
\pagestyle{headings}

% adjust as needed
\addtolength{\footskip}{0\baselineskip}
\addtolength{\textheight}{-1\baselineskip}

%define colors for code language
\definecolor{codegreen}{rgb}{0,0.7,0.3}
\definecolor{codegray}{rgb}{0,0,0}
\definecolor{codepurple}{rgb}{0.58,0,0.82}
\definecolor{backcolour}{rgb}{0.95,0.95,0.95}
\definecolor{keywordcolor}{rgb}{0.8,0.3,0}

\lstdefinestyle{mystyle}{
    backgroundcolor=\color{backcolour},   
    commentstyle=\color{codegreen},
    keywordstyle=\color{keywordcolor},
    numberstyle=\tiny\color{codegray},
    stringstyle=\color{codepurple},
    basicstyle=\ttfamily\footnotesize,
    breakatwhitespace=false,         
    breaklines=true,                 
    captionpos=b,                    
    keepspaces=true,                 
    numbers=left,                    
    numbersep=5pt,                  
    showspaces=false,                
    showstringspaces=false,
    showtabs=false,                  
    tabsize=4
}

\lstset{style=mystyle}



%%%%%%%%%%%%%%%%%%%%%%%%%%%%%%%%
%%%%% INIZIO DEL DOCUMENTO %%%%%
%%%%%%%%%%%%%%%%%%%%%%%%%%%%%%%%
\begin{document}



%%%%%%%%%%%%%%%%%%%%%%%%%%%%
%%% TITOLO DEL DOCUMENTO %%%
%%%%%%%%%%%%%%%%%%%%%%%%%%%%
\title{Programmazione di Sistema e di Rete}



%%%%%%%%%%%%%%%%%%%%%%%%%%
%%%%%%%%% AUTORE %%%%%%%%%
%%%%%%%%%%%%%%%%%%%%%%%%%%
\author{Matteo Aprile\\
				Professore: Franco Tommasi\\
        }
        
%scrive il titolo
\maketitle

%scrive l'indice
\tableofcontents
\underline{\hspace{ 80 mm }}



%%%%%%%%%%%%%%%%%%%%%%%%%%%%%
%%% SEZIONI DEL DOCUMENTO %%%
%%%%%%%%%%%%%%%%%%%%%%%%%%%%%
\input{sections/0_libri/0_libri}
\newpage
\section{Comandi utili}

% find
\subsection{find: trovare tutti i file eseguibili}
	
\begin{lstlisting}
$ find . -type f -perm -0100
./standards/makeopt.awk
./standards/makeconf.awk
./proc/awkexample
./systype.sh
./advio/fixup.awk
\end{lstlisting}


\subsection{find: trovare file di intestazione del mac come stdio.h}

\begin{lstlisting}
$ find /Applications/Xcode.app/ -name stdio.h 2>/dev/null
\end{lstlisting}


% ldd
\subsection{lld: per capire che librerie usa il codice}

\begin{lstlisting}
$ ldd [nomevodice]
\end{lstlisting}


% gcc
\subsection{gcc: per vedere tutta la gerarchia di file in una libreria}

\begin{lstlisting}
$ gcc -H lib.a
\end{lstlisting}


\subsection{gcc: per vedere il codice con tutti i file importati}

\begin{lstlisting}
$ gcc -E file.c
\end{lstlisting}


\subsection{gcc -g: debugging debole}

\begin{lstlisting}
$ gcc -g -ansi -I../include -Wall -DMACOS -D_DARWIN_C_SOURCE  ls1.c -o ls1  -L../lib -lapue
\end{lstlisting}


\subsection{gcc -ggbd: debugging forte}

\begin{lstlisting}
$ gcc -ggbd -ansi -I../include -Wall -DMACOS -D_DARWIN_C_SOURCE  ls1.c -o ls1  -L../lib -lapue
\end{lstlisting}


% xattr
\subsection{xattr: usato per i file che entrano in quarantena su MacOS}

\begin{lstlisting}
$ xattr -d (delete) com.apple.quarantine [path sh]
\end{lstlisting}





\section{Variabili di sistema definite in .bashrc}

% INC
\subsection{INC}

\begin{lstlisting}
INC="/Applications/Xcode.app/Contents/Developer/Platforms/MacOSX.platform/Developer/SDKs/MacOSX.sdk/usr/include/"
\end{lstlisting}


\newpage
\section{Introduzione - 23/27.09.22}

% System call
\subsection{System call}

Sono \hl{uguali alle funzioni di libreria dal punto di vista sintattico, pero' cambia il modo di compilarle}. Notare che non possono essere usati i nomi delle SC per delle function call.

Per poi poter "raccontare" tra umani le sequenze di bit che vengono mandate ai processori si usa \hl{assembly}.

Sono effettivamente delle chiamate a funzioni ma poi dal codice assembly puoi capire che è una system call dato che ha dei meccanismi specifici.

Alcuni esempi di chiamate e registri:
\begin{itemize}
	\item \textbf{eax} : registro dove metti il \textbf{numero della sc}
	\item \textbf{int 0x80}: \textbf{avvisa il kernel} che serve chiamare una sc
	\item \textbf{exit()}: chiudere un processo
	\item \textbf{write()}:
\end{itemize}

\begin{lstlisting}
mov edx,4    ; lunghezza messaggio
mov ecx,msg  ; puntatore al messaggio
mov ebx,1    ; file descriptor
mov eax,4    ; numero della sc
int 0x80	
\end{lstlisting}

dove nel \hl{file descriptor} indichi a quale file devi mandare l'output. Questo viene usato dato che così non deve cercare il path ogni volta ma lo mantiene aperto riferendosi ad esso tramite il numero.


% Programma Make
\subsection{Programma Make - 30.09.22}

Quando viene avviato verifica la presenza di un file chiamato "Makefile", oppure si usa 'make -f'. In questo file ci sono le \hl{regole di cosa fare per automatizzare delle azioni per un numero n di file}. Se, durante la compilazione di massa, \hl{una di queste da un errore il programma make si interrompe}, per evitare ciò si usa '-i' (ignore).

Il Makefile andrà ad aggiornare una libreria andando a guardare se una delle 3 date di ultima modifica si sono aggiornate.


Andiamo a guardare \hl{cosa contiene Makefile}:

\begin{lstlisting}
DIRS = lib intro sockets advio daemons datafiles db environ \
	fileio filedir ipc1 ipc2 proc pty relation signals standards \
	stdio termios threadctl threads printer exercises

all:
	for i in $(DIRS); do \
		(cd $$i && echo "making $$i" && $(MAKE) ) || exit 1; \
	done

clean:
	for i in $(DIRS); do \
		(cd $$i && echo "cleaning $$i" && $(MAKE) clean) || exit 1; \
	done
\end{lstlisting}

dove:
\begin{itemize}
	\item \hl{DIRS}: lo si associa alle \textbf{stringhe singole} che gli sono state associate
	\item \hl{all}: nel ciclo for:
		\begin{itemize}
			\item manda un comando in subshell
			\item \$\$i: riferimento alla variabile "i" del for + simbolo escape per il Makefile
			\item \$(MAKE): macro predefinita per i Makefile
		\end{itemize}
 
\end{itemize}

La struttura è:

\begin{lstlisting}
target: prerequisiti
	rule
\end{lstlisting}

dove:

\begin{itemize}
	\item \hl{target}: è l\textbf{a cosa che si vuole fare}, se essendo il primo target, sarà anche quello di default
	\item \hl{prerequisiti}: \textbf{file e/o target} a loro volta
	\item \hl{rule}: indica \textbf{cosa puo' fare il target}
\end{itemize}

Può capitare che prima di eseguire il Makefile ci sia uno \hl{script "configure"}. 

In molti casi si ha un \hl{target "clean"} che permette di pulire i file .o che sono inutili dopo la compilazione, o comunque qualsiasi tipo di file gli si voglia far eliminare. Questo tipo di target che non rappresentano un file, sono detti \hl{"phony"} perchè fasulli, dato che non sono file ma sole parole

\begin{lstlisting}
file: file.o lib.o

clean:
	rm file.o
\end{lstlisting}

Abbiamo delle \hl{variabili automatiche} per rendere il lavoro più facile:

\begin{itemize}
	\item \textbf{\$@}: per riferirsi il target
	\item \textbf{\$?}: tutti i prerequisiti più recenti del target
	\item \textbf{\$\^}: tutti i prerequisiti del target
	\item \url{https://www.gnu.org/software/make/manual/make.html#Automatic-Variables}
\end{itemize}


Un altro esempio di Makefile è:

\begin{lstlisting}
ROOT=..
PLATFORM=$(shell $(ROOT)/systype.sh)
include $(ROOT)/Make.defines.$(PLATFORM)

PROGS =	getcputc hello ls1 mycat shell1 shell2 testerror uidgid

all:	$(PROGS)

%:	%.c $(LIBAPUE)
	$(CC) $(CFLAGS) $@.c -o $@ $(LDFLAGS) $(LDLIBS)

clean:
	rm -f $(PROGS) $(TEMPFILES) *.o

include $(ROOT)/Make.libapue.inc
\end{lstlisting}

dove:

\begin{itemize}
	\item ROOT: cwd
	\item PLATFORM: assumera in valore del OS: macos/linux
	\item include: include un file
	\item PROGS: elenco dei programmi da usare
	\item \%: target con nome variabile, indica un file
	\item \%.c: target con nome variabile ma estensione .c
	\item \$(CC): indica il compilatore dove cc è un link simbolico a clang
	\item \$(CFLAGS) indica una macro predefinita dei default vuota che si può usare all'occorrenza
	\item all: target che prende in carico tutti i programmi che se saranno di tipo .c saranno presi in carico dal target successivo
\end{itemize}


\hl{Per la compilazione dei file}, qualsiasi sia il linguaggio, \hl{make sapra' come compilarlo} grazie a tutte le definizioni di default presenti in:

\begin{lstlisting}
make -p
\end{lstlisting}

notare che \textbf{si puo' mettere un comando custom nelle rule} del target

Nell'eventualità di \hl{voler aggiornare un solo file della libreria senza far aggiornare il resto ci bastera' usare uno script} che compila quel file passato a linea di comando.

\begin{lstlisting}
gcc -ansi -I../include -Wall -DMACOS -D_DARWIN_C_SOURCE  ${1}.c -o ${1}  -L../lib -lapue
\end{lstlisting}



% Direttive di preprocessore
\subsection{Direttive di preprocessore - 28.09.22}
Sono delle \hl{indicazioni date a gcc prima di iniziare la compilazione}.

Iniziano tutte con '\#':
\begin{itemize}
	\item \textbf{\#include}: serve ad \textbf{includere delle librerie} di sistema ($<$lib.h$>$) oppure di librerie fatte da noi e non in directory standard ("lib.h")
	\item \textbf{\#define}:
		\begin{itemize}
			\item permette di \textbf{creare delle "macro"}, che vanno a sostituire una stringa con un'altra (es: \#define BUFLEN), può capitare che debbano essere definite delle macro prima che si compili il programma, in questi casi si usa scrivere es: '-DMACOS'
			\item permette di \textbf{creare delle "function like macro"} (es: \#define ABSOLUTE\_VALUE(x) (((x$<$0)?-(x):(x))  
		\end{itemize}
		
	\item \textbf{\#ifdef, \#ifndef, \#endif}: usata per far accadere qualcosa nel caso un macro sia stata definita
\begin{lstlisting}
#ifdef VAR
print("hello");
#endif
\end{lstlisting}

\end{itemize} 

\hl{Per evitare che piu' file includano lo stesso si usano degli \#ifndef} in tutto il codice, in modo da evitare doppie definizioni.


% Librerie
\subsection{Librerie}

Durante la fase di compilazione creiamo dei file oggetto (.o) per ogni file in cui è scritta la descrizione delle funzioni di libreria (.c)

\begin{lstlisting}
gcc -c bill.c
\end{lstlisting}

Si andrà poi a creare il \hl{prototipo della funzione (.h)}.

In fine \hl{tramite il linker si andranno ad unire tutti i file per crearne uno unico} con tutte le definizioni delle funzioni incluse nelle librerie, di sistema e non, importate. Si vanno quindi a \hl{sciogliere tutti i riferimenti incrociati}.

\begin{lstlisting}
	gcc -o program program.o bill.o
\end{lstlisting}

Per quanto riguarda le \hl{funzioni di sistema} NON abbiamo il file sorgente ma abbiamo direttamente l'eseguibile. In compenso abbiamo un \hl{file di libreria}, cioè un insieme di file oggetto linkati in un unico file, dove c'è il codice oggetto di tutte le funzioni.

Abbiamo \textbf{2 tipi di librerie}:
\begin{itemize}
	\item \hl{statiche}: è una \textbf{collezione di file oggetto} che hanno il codice compilato delle funzioni e che verranno \textbf{linkati al momento della compilazione}. Il programma che si crea sarà possibile essere eseguito solo sullo stesso OS.
		
		Il \textbf{problema si ha nell'aggiornamento delle librerie al momento della scoperta di un bug}. Una volta coretto servirà ricevere la versione corretta per poter aggiornare il programma.
		
	\item \hl{dynamic}: ricordano il concetto di plug-in, quindi \textbf{viene invocato a runtime e caricato in memoria} (es: aggiornamenti dei OS). \textbf{L'eseguibile non viene toccato la correzione avviene solo nella libreria}.
	
		Il requisito maggiore è che chi si passa il codice debba avere lo stesso OS dell'altro utente. Notare che \textbf{non cambia il prototipo} dato che sennò bisognerà ricompilare l'intero programma.
\end{itemize}


In generale le \hl{librerie statiche sono molto pericolose} infatti alcuni OS le aboliscono \hl{per le questioni di sistema}. Su linux si ha come libreria statica 'lib.c' che è la libreria con le funzioni più usate in c. Per macos è stata abolita.

Per compilare con la versione dinamica non servono opzioni, per la statica si usa:

\begin{lstlisting}
gcc -static
\end{lstlisting}


% Creazione librerie
\subsection{Creazione librerie}

Per costruire una \hl{libreria statica per MacOS}:

\begin{enumerate}
	\item costruiamo il \textbf{file oggetto}:

\begin{lstlisting}
gcc -c libprova.c
\end{lstlisting}
		
	\item costruiamo la \textbf{libreria} (con ar=archive, c=create se lib.a non esiste):

\begin{lstlisting}
ar rcs libprova.a libprova.o
\end{lstlisting}
			
	\item costruire il \textbf{codice} che usa la libreria (con -Wall=verbose warning, -g=debugging, -c=create del file):

\begin{lstlisting}
gcc -Wall -g -c useprova.c
\end{lstlisting}
	
	\item \textbf{linker} per risolve le chiamate incrociate (con -L.=dove prendere la libreira, -l[nomelib]=usare la libreria):

\begin{lstlisting}
gcc -g -o useprova useprova.o -L. -lprova 
\end{lstlisting}

\end{enumerate}

Per capire che librerie usa il codice si usa:

\begin{lstlisting}
otool -L [nomecodice]
\end{lstlisting}


Per costruire una \hl{libreria statica per Linux}:

\begin{enumerate}
	\item costruiamo il \textbf{file oggetto}:

\begin{lstlisting}
gcc -fPIC -Wall -g -c libprova.c
\end{lstlisting}

	\item costruiamo la \textbf{libreria} (con 0.0=versione della libreira):

\begin{lstlisting}
gcc -g -shared -Wl,-soname,libprova.so.0 -o libprova.so.0.0 libprova.o -lc 
\end{lstlisting}

	\item costruire il \textbf{link simbolico per aggiornare le librerie} senza aggiornare gli eseguibili e senza cambiare il nome del programma:

\begin{lstlisting}
ln -sf libprova.so.0.0 libprova.so.0 
\end{lstlisting}

	\item \textbf{linker} per risolve le chiamate:

\begin{lstlisting}
ln -sf libprova.so.0 libprova.so
\end{lstlisting}

\end{enumerate}


Per capire che librerie usa il codice si usa:

\begin{lstlisting}
ldd [nomevodice]
\end{lstlisting}


% Aggiornamento librerie
\subsection{Aggiornamento librerie}

Su \textbf{Linux} il sistema \hl{andra' a prendere direttamente una libreria dinamica}, per evitare ciò e far trovare la nostra, basterà \hl{impostare una variabile di ambiente}:

\begin{lstlisting}
LD_LIBRARY_PATH=`pwd` ldd useprova
\end{lstlisting}

Tipicamente la libreria viene distribuita nelle directory di sistema andandola ad "installare".


Su \textbf{MacOS} la libreria dinamica è un \hl{.dylib}:

\begin{lstlisting}
gcc -dynamiclib libprova.c -o libprova.dylib
\end{lstlisting}

Quindi eseguendo il programma \hl{trovera' la libreria controllando nella directory corrente} e quindi non serve creare la variabile di ambiente come su Linux.

i file di intestazione del mac come stdio.h per cercarla uso:

\begin{lstlisting}
find /Applications/Xcode.app/ -name stdio.h 2>/dev/null
\end{lstlisting}



\newpage
\section{System call}

% Funzioni e system call
\subsection{Funzioni e system call}

Se prendiamo un funzionamento più semplice del comando "ls" potrebbe essere:

\begin{lstlisting}
#include "apue.h"
#include <dirent.h>

int
main(int argc, char *argv[])
{
	DIR				*dp;
	struct dirent	*dirp;

	if (argc != 2)
		err_quit("usage: ls1 directory_name");

	if ((dp = opendir(argv[1])) == NULL)
		err_sys("can't open %s", argv[1]);
	while ((dirp = readdir(dp)) != NULL)
		printf("%s\n", dirp->d_name);

	closedir(dp);
	exit(0);
}
\end{lstlisting}


dove abbiamo che:

\begin{itemize}
	\item \hl{DIR}: ??
	\item \hl{struct dirent}: \textbf{tipo struttura} che contiene al suo interno diversi tipi di variabili. 
	
		Per capire se è una funzione di sistema lanciamo:
		
\begin{lstlisting}
grep -rw "struct dirent" $INC
\end{lstlisting}
		
		seguiamo il percorso:
		
\begin{lstlisting}
/Applications/Xcode.app/Contents/Developer/Platforms/MacOSX.platform/Developer/SDKs/MacOSX.sdk/usr/include//sys/dirent.h:struct dirent {
\end{lstlisting}
		
\begin{lstlisting}
#ifndef _SYS_DIRENT_H
#define _SYS_DIRENT_H

#include <sys/_types.h>
#include <sys/cdefs.h>

#include <sys/_types/_ino_t.h>


#define __DARWIN_MAXNAMLEN      255

#pragma pack(4)

#if !__DARWIN_64_BIT_INO_T
struct dirent {
	ino_t d_ino;                    /* file number of entry */
	__uint16_t d_reclen;            /* length of this record */
	__uint8_t  d_type;              /* file type, see below */
	__uint8_t  d_namlen;            /* length of string in d_name */
	char d_name[__DARWIN_MAXNAMLEN + 1];    /* name must be no longer than this */
};
#endif /* !__DARWIN_64_BIT_INO_T */

#pragma pack()

#define __DARWIN_MAXPATHLEN     1024

#define __DARWIN_STRUCT_DIRENTRY { \
	__uint64_t  d_ino;      /* file number of entry */ \
	__uint64_t  d_seekoff;  /* seek offset (optional, used by servers) */ \
	__uint16_t  d_reclen;   /* length of this record */ \
	__uint16_t  d_namlen;   /* length of string in d_name */ \
	__uint8_t   d_type;     /* file type, see below */ \
	char      d_name[__DARWIN_MAXPATHLEN]; /* entry name (up to MAXPATHLEN bytes) */ \
}

#if __DARWIN_64_BIT_INO_T
struct dirent __DARWIN_STRUCT_DIRENTRY;
#endif /* __DARWIN_64_BIT_INO_T */



#if !defined(_POSIX_C_SOURCE) || defined(_DARWIN_C_SOURCE)
#define d_fileno        d_ino           /* backward compatibility */
#define MAXNAMLEN       __DARWIN_MAXNAMLEN
/*
 * File types
 */
#define DT_UNKNOWN       0
#define DT_FIFO          1
#define DT_CHR           2
#define DT_DIR           4
#define DT_BLK           6
#define DT_REG           8
#define DT_LNK          10
#define DT_SOCK         12
#define DT_WHT          14

/*
 * Convert between stat structure types and directory types.
 */
#define IFTODT(mode)    (((mode) & 0170000) >> 12)
#define DTTOIF(dirtype) ((dirtype) << 12)
#endif


#endif /* _SYS_DIRENT_H  */
\end{lstlisting}

		dove vediamo che se la variabile "\_\_DARWIN\_64\_BIT\_INO\_T" è stata definita avremo che la struttura di struct dirent è:
		
\begin{lstlisting}
#define __DARWIN_STRUCT_DIRENTRY { \
	__uint64_t  d_ino;      /* file number of entry */ \
	__uint64_t  d_seekoff;  /* seek offset (optional, used by servers) */ \
	__uint16_t  d_reclen;   /* length of this record */ \
	__uint16_t  d_namlen;   /* length of string in d_name */ \
	__uint8_t   d_type;     /* file type, see below */ \
	char      d_name[__DARWIN_MAXPATHLEN]; /* entry name (up to MAXPATHLEN bytes) */ \
}

#if __DARWIN_64_BIT_INO_T
struct dirent __DARWIN_STRUCT_DIRENTRY;
#endif /* __DARWIN_64_BIT_INO_T */
\end{lstlisting}

		
	\item \hl{if}: esegue un controllo sugli args. Notiamo che "err\_quit" non è una funzione di sistema da:
	
\begin{lstlisting}
grep -rw "err_quit" $INC
\end{lstlisting}

		infatti non restituisce nulla. Deve allora essere una funzione di libreria create da noi quindi non presente nella directory standard.
		
		La funzione andrà a dare un messaggio di errore e poi esce dal programma.
		
		
	\item \hl{opendir}: serve ad aprire una directory andandola a caricare nella RAM. 
	
	\item \hl{while}: leggiamo la directory e la inseriamo nella struttura che poi sarà richiamata tramite:
	
\begin{lstlisting}
dirp->d_name
\end{lstlisting}

		dove "d\_name" è il nome dello slot in cui è contenuto il nome del file.
		
		
	\item \hl{exit}: restituisce l'exit code del programma
	
\end{itemize}


% Capire se una funzione è una system call
\subsection{Capire se una funzione è una system call}

Andiamo a \hl{vedere se e' una funzione o una system call tramite "man"}, lo si capisce tramite la dicitura in alto alla pagina del manuale:

	\begin{itemize}
		\item \textbf{Library Functions} Manual
		\item \textbf{System Calls} Manual
	\end{itemize}

Abbiamo anche \hl{esempi piu' particolari}, come fork, dove è indicata come system call ma in realtà le richiama ma in prima persona.


Potremo trovare i simboli di una libreria tramite:

\begin{lstlisting}
nm lib.a
\end{lstlisting}

che ci fa vedere, per ogni file oggetto, i simboli associati per ogni funzione.

Le system call le troveremo in "\$INC/sys/syscall.h"

\newpage
\section{Gli standard}

% Storia e basi
\subsection{Storia e basi}

La \hl{standardizzazione di UNIX e' iniziata nel 1988} facendo affidamento ad alcuni \hl{standard di C} dato che fa usi di interfaccie e prototipi.

In definitiva abbiamo gli standard:

\begin{itemize}
	\item Posix.1-2001 / SUSv3: (\url{http://pubs.opengroup.org/onlinepubs/009604599/})
	\item Posix.1-2008 / SUSv4:  più usato in ambiti di automazioni aziendali, infatti sono specializzate sullo scambio di informazione in segnali realtime. Per questo la sua certificazione non è stata presa da nessuno se non fa un IBM. (\url{http://pubs.opengroup.org/onlinepubs/9699919799/})
\end{itemize}

(\hl{PS: le versioni sono back compatibili} quindi se settiamo -D\_XOPEN\_SOURCE=700 non precludiamo la SUSv3)

Nonostante gli standard \hl{ogni OS fa delle sue modifiche su alcune cose esterne alle SUS}.

Per verificare il tipo di standard su un applicativo (\_XOPEN\_SOURCE) o un sistema (\_XOPEN\_VERSION), si fa affidamento alle "\hl{feature test macros}" consultabili dai \textbf{codici di intestazione .h}. Per esempio \_XOPEN\_SOURCE impostata a 600 o 700 indica SUSv3 o SUSv4.

\begin{lstlisting}
-D_XOPEN_SOURCE=600
\end{lstlisting}

in questo modo potremo allora andare a compilare tutti i programmi conformi su qualsiasi OS.


% Limiti
\subsection{Limiti}

Abbiamo dei \hl{limiti di compilazione} che possono essere visti nei file di intestazione.

Possiamo visualizzare i runtime limit che tramite le funzioni: sysconf (es: lunghezza massima del nome dei file che dipende dal filesystem può capirlo tramite pathconf su un file qualunque di quel filesystem)

\begin{itemize}
	\item \hl{sysconf}: usato per \textbf{determinare il valore corrente di un limite}
	
	\item \hl{pathconf}: da \textbf{informazioni sul file system} e per fare ciò gli serve poter arrivare ad un qualunque file del filesystem
	
	\item \hl{fpathconf}: come pathconf ma \textbf{prende il file descriptor}
\end{itemize}

Tutti e 3 prendono come \hl{parametro}:

\begin{lstlisting}
int name
\end{lstlisting}

che \hl{restituisce una chiave} in base a cosa si vuole indagare. In pratica \hl{fanno riferimento ad un nome simbolico che si riferisce ad un valore}.
Tutte queste chiamate fanno si di avere più \textbf{portabilta'}. Se queste chiamate sono fatte da file include \hl{vincono sempre quelli di sysconf. Saranno precedute da \_SC\_ per i sysconf e da \_PC\_ per i pathconf.


% Determinare l'allocazione
\subsection{Determinare l'allocazione}

Supponendo di avere \hl{bisogno di uno spazio} dove mettere un nome di file (path) per poterlo gestire. Per capire quanto spazio devo allocare si utilizza una funzione "path\_alloc". Questa funzione \hl{retituisce un puntatore ad una memoria capace di contenere il massimo dei caratteri}.

Per vedere qual'è la lunghezza usiamo la variabile limite:

\begin{lstlisting}
pathconf(_PC_NAME_MAX)
\end{lstlisting}

in genere avremo NAME\_MAX = 255.


% Definizione di un tipo

pid\_t sono definiti così dato che il progettista vuole lasciare libero il prgrammatore dal tipo 

HOMEWORK: quanto vale?

\begin{lstlisting}
$ grep -rw "pid_t" $INC | grep typedef

/sys/_types/_pid_t.h:typedef __darwin_pid_t pid_t;
/sys/_types.h:typedef __uint32_t __darwin_id_t;

$ grep -rw "__uint32_t" $INC | grep typedef

/i386/_types.h:typedef unsigned int __uint32_t;
\end{lstlisting}

tutti questi rimandi sono dati dalla \hl{portabilita'}



NUOVI CAPITOLO: file IO

in file io co sono le funzioni che fanno il buffered io che gestisce lui e si contrappone da quello dello stdlib. 

chiamata open():

1 arg: path che può essere dato come assoluto o relativo 

2: flag: sono dei bit che dicono cosa fare (es: modalità append)

notare che per le read non appena fatte le scritture il file continua a leggere da dove è stata l'ultima "posizione del file" (current posizion) della read

la prossia write sarà alla fine della read. all'inizio sta a 0 inizo file

avremo che si metterà ad 1 il bit del flag che ci serve tramite:

\begin{lstlisting}
open(file, O_RDWR | O_APPEND | O_CREAT | O_TRUNC, file_mode)
\end{lstlisting}

avremo allora: $11000001010$

con O\_RDWR: 2, O\_APPEND: 8, O\_CREAT: 512, O\_TRUNC: 1024

per lettura e scrittura invece ...



possono essere 3 argomenti quando i file lo stai creando

3. mode: privilegi con cui i file deve essere creato

openat():
si prende il file descriptor di una directory e poi passare un path che viene interpretato con un path relatico a quella directory. quidni ogni cosa viene fatta in questa orecotry anche se nelle directory a monte non si hanno i privilegi

quindi andiamo ad usare open() sulla direcory per avere il file descriptor da usare in openat()



\newpage
\section{File I/O}


% Introduzione
\subsection{Chiamata open}

I file I/O sono le \hl{funzioni che gestisce buffered I/O} ed in contrapposizione da quelle della librerira "stdlib". La chiamata open() fa parte di queste funzioni, i suoi argomenti sono:

\begin{itemize}
	\item \hl{arg}: path assoluto o relativo
	\item \hl{flag}: bit che indica l'\textbf{attivazione di alcune modalita'}. Si avrà allora a \textbf{settare il bit della flag a 1}:
	\item \hl{mode}: serve a dare i \textbf{privilegi con cui i file deve essere creato} (\textbf{da usare solo nella creazione del file})
\begin{lstlisting}
open(file, O_RDWR | O_APPEND | O_CREAT | O_TRUNC, file_mode)
\end{lstlisting}

		avremo allora: $11000001010$

		con O\_RDWR: 2, O\_APPEND: 8, O\_CREAT: 512, O\_TRUNC: 1024
		
		
\end{itemize}

Appena \hl{aperto il file questo avra' la "current position" a 0}, cioè ad inizio file. Iniziando a scrivere avremo che alla prima lettura la nostra current position si sarà spostata fino a dopo abbiamo finito di scrivere.

Per quanto riguarda \hl{read e write i bit delle flag} abbiamo un modo "scomodo":

\begin{lstlisting}
#define O_RDONLY        0x0000          /* open for reading only */
#define O_WRONLY        0x0001          /* open for writing only */
#define O_RDWR          0x0002          /* open for reading and writing */
#define O_ACCMODE       0x0003          /* mask for above modes */
\end{lstlisting}

sono dette quindi \hl{maschere} per leggere o scrivere.

Una variante e' \hl{openat()} che prende il file descriptor (passato dalla open() sulla directory) di una directory per passare il path relativo a quella directory. Abbiamo quindi la \hl{possibilita' di accedere a delle directory anceh senza avere i privilegi}. ???

il permesso di esecuzione delle directory serve a cercare un nome in una directory. se il permesso è disattivato non si riesce ad arrivare alla foglia, se invece apri una directory il sistema sa già a quale inod andare dato che si salva quello al posto di tutto il path. quindi fai diventare relativi i privilegi per quella directory. 

questa chiamata è interesante per quando le thread vogliono lavorarein un loro ambiente.

alla openat è associata una flag alla firectory aperta dalla open, allora usiamo la flag O\_DIRECTORY. nelle flag vediamo la differnza tra UNIX e gli standard dato che ce ne sono un certo numero di base e poi ogni UNIX ne agiunge altri. Gli standard gli troviamo nel manuale del SUSv3 e SUSv4.

O\_CREATE: flag per dire che si vuole creare un file. una trappola che c'è è la builtin "umask" usata per dare dei permessi di base. questa umask è presente in ogni processo e la eradita dal padre (ma può comunque modificarla). ci restituisce un numero ottale (perche inizia xon 0) e dice di togliere la scrittura ai gruppi

\begin{lstlisting}
$ umask
0022

$ ll file
-rw-r--r-- 1 docente staff 0 14 Ott 09:02 file
\end{lstlisting}

quindi la umask taglia i permessi dei file. quidndi se diamo un 666 nella creazione di un file con open ci ritroveremo 644.

altra trappola: se creo un file in sola scrittura e lettura ecc, potremo usare una chiamata per tolgiere dei flag in corsa. 


un altra trappola: la cosa strana è che se apri un file in rw e ottieni i privilegi, se i privilegi vengono cambiati dalla linea di comando togliendo w se vado a scrivere in quel file perchè aperto con la open in precedenza, quel file comunque puoi continuare a scriverlo dato che il permesso non viene revocato


O\_TRUNC: nella cfreazione dell file se il file esiste ed è aprto per rw viene azzerato il file (se vuoi aprire un file nuovo ed essite già il veccjhio viene azzerato)

O\_EXCL : nella creazione del file se in file già esiste fa fallire la chiamata. qundo si fa una read potrebbe essere appesa per giorni, ma se il file viene aperto con questo file, se non c'è nula da leggere o scrivere ritorna e non le lascia appese

chaiamata lseek: in un dato momento se hai un file descrptior sta in un certo punto del file (se non lo hai toccato stai all'inizio). lseek sui file già aperti permette di spostare la "current position" 

1 arg: file descriptor
2. offset per dire dove ci si vuole spostare
3. whence: ("da dove") puo essere: SEEK\_SET valore preciso di dove andare. SEEK\_CUR può ssere neagativo ed indica il gap dopo il quale nadare cioe di quanto sposarsi, SEEK\_END valore rispetto alla file sarà negativo ma se metto valore positivo creun un uco dato che salto alcuni blocchi. nell'inod viene indicato il file buchi compresi, l'inod ricrodiamo che ha 256 byte dove un pò sono usati per i metadati e nel resto abbiamo dei puntatori a dei blocchi che sono a loro volta blocchi di puntatori ecc quidni non c'è il concetto di buco coi blocchi perchè semplicemente ci son u può di blochci da una parte e un pò dall'altra 

Un esempio di buco lo creiamo:

\begin{lstlisting}
#include "apue.h"
#include <fcntl.h>

char	buf1[] = "abcdefghij";
char	buf2[] = "ABCDEFGHIJ";

int
main(void)
{
	int		fd;

	if ((fd = creat("file.hole", FILE_MODE)) < 0)
		err_sys("creat error");

	if (write(fd, buf1, 10) != 10)
		err_sys("buf1 write error");
	/* offset now = 10 */

	if (lseek(fd, 16384, SEEK_SET) == -1)
		err_sys("lseek error");
	/* offset now = 16384 */

	if (write(fd, buf2, 10) != 10)
		err_sys("buf2 write error");
	/* offset now = 16394 */

	exit(0);
}	
\end{lstlisting}

avremo:

\begin{lstlisting}
$ xxd file.hole
00000000: 6162 6364 6566 6768 696a 0000 0000 0000  abcdefghij......
00000010: 0000 0000 0000 0000 0000 0000 0000 0000  ................
00000020: 0000 0000 0000 0000 0000 0000 0000 0000  ................
00000030: 0000 0000 0000 0000 0000 0000 0000 0000  ................
00000040: 0000 0000 0000 0000 0000 0000 0000 0000  ................
00000050: 0000 0000 0000 0000 0000 0000 0000 0000  ................
00000060: 0000 0000 0000 0000 0000 0000 0000 0000  ................
00000070: 4142 4344 4546 4748 494a                 ABCDEFGHIJ
\end{lstlisting}

abbiamo che la memoria effettiva sul disco è:

\begin{lstlisting}
$ du file.hole
3208	file.hole
\end{lstlisting}

invece la size del file viene dichiarata:

\begin{lstlisting}
$ stat -x file.hole 
  File: "file.hole"
  Size: 1638410      FileType: Regular File
  Mode: (0644/-rw-r--r--)         Uid: (  501/    matt)  Gid: (   20/   staff)
Device: 1,16   Inode: 27657406    Links: 1
Access: Fri Oct 14 09:59:17 2022
Modify: Fri Oct 14 09:57:59 2022
Change: Fri Oct 14 09:57:59 2022
 Birth: Fri Oct 14 09:47:24 2022
\end{lstlisting}




esistono dei file descriptor seekble e non che cerca di capire se è un file regolare o meno:

\begin{lstlisting}
#include "apue.h"

int
main(void)
{
	if (lseek(STDIN_FILENO, 0, SEEK_CUR) == -1)
		printf("cannot seek\n");
	else
		printf("seek OK\n");
	exit(0);
}
\end{lstlisting}




ci sono piu casi in cui il numero di byte letto è minore di quello chiesto:
1. possiamo avere un valore di ritorno di 50 se chiedo di leggere 100 perchè ci sono solo 50 byte
2. quando legge da un terminale: la read ritorna quando dai invio
3. quando legge da una rete: puoi leggere 100 byte ma se ne leggi 10 puoi gestire tu cosa fare, in genere se non arriva nulla la read rimane in attesa e non gli arrivano byte rimane appesa sennò se son oarrivati rimane appesa e poi interrompe quando non sente più nulla
4. quando legge da una pipe: se non arriva nulla rimane appesa, se arriva qualcosa interrompe e restituisce quello che ha letto
5. quando interrotta da un segnale e alcuni dati sono stati letti gli restituisce



\end{document}
%%%%%%%%%%%%%%%%%%%%%%%%%%%%%%%%
%%%%%% FINE DEL DOCUMENTO %%%%%%
%%%%%%%%%%%%%%%%%%%%%%%%%%%%%%%%




