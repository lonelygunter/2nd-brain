\newpage
\section{Network Flow Model}


% Introduzione
\subsection{Introduzione}
dove abbimao un grafo  con dei sui parametri che assieme formano un network. dobbiamo allora capire con=me un flusso di dati transitra nellarete.
primo porblema che vediamo
PROBLEMA DI FLUSSO A COSTO MINIMO:

ha la varainte lineare
variante single commodity


in questa classe di problemi abbiamo un grafo G(V, A)
con v vertici e A archi

img


inq uestso caso ogni vetice $i \in V$ o generea o assorbe un flusso di materiali, dati, ecc...

parametro per i nodi:
questo $\forall i \in V: d_i$
> 0 per i = sorgente (souce)
= 0 per i = transito
< 0 per i = pozzo (sink)


se d_i > 0 indica un afornitura
< 0 indica una domanda

caristiche:
single commodity: flusso omogeneo
multiple comodity: varie tipologie di flusso 


parametri per gli archi
prendo un parametro c_{ij} cioe un arco che colega i nodi i e j. c_{ij} = costo (di traspost) unitario (per unita di flusso)
altro paramentro q_{ij}: capacita massima (qunato flusso puo transitrar da i a j)


decisione da prendere: allocaizone del flusso, cioe qunto flusso deve transitare su ogni arco
le varaibili saranno: x_{ij} >= 0 fluso che transita da i a j per unita di tempo


affinche si possa avere una soluzione ammissibile una condizione necessaria è che se sommiamo su utti i vertici le quntita d_i devo avere 0:
$$\sum_{i\in V} d_i = 0$$

quindi non sara sufficiente per via delle capacita degli archi



misure di prestazione della f.o.:
vigliamo allora minimizzare il costo totale di trasporto su tutti gli archi:
$$\min \sum_{(i,j)\in A} c_{ij} x_{ij}$$

notaione epr definire un insieme di vincoli:
dato uun nodo i al quale abbiamo archi uscenti ed entranti allora tutti li archi entranti associamo un insieme $\delta^-(i)$ per quellio uscenti: $\delta^+(i)$


definiiamo allora i vincoli, dobbiamo tenere in conto dei vincoli di conservazione del flusso: cioe che dtutto quello cheentra nel nodo deve uscirci:
$$\sum_{(i,j)\in\delta^+(i)} x_{ij} - \sum_{(j,i)\in\delta^-(i)} x_{ji} = d_i\ \ \ \forall\ \ \ i\in V$$

altri vincoli per avere una soluzione ammissibile abbiamo ance dei vincoli di capacita: avremo un vincolo associato ad ogni arco:
$$x_{ij} \leq q_{ij}\ \  \forall\ \ \ i,j\in A$$

vincoli di non negativita:
$$x_{ij} \geq 0 (\geq l_i)\ \ \ \forall\ \ \ i,j\in A$$

dove l_i quantita minimia in caso di evenienza

la formulazione compatta dice:
$$\min z=\sum_{(i,j)\in A} c_{ij} x_{ij}$$
s.t
- $\sum_{(i,j)\in\delta^+(i)} x_{ij} - \sum_{(j,i)\in\delta^-(i)} x_{ji} = d_i\ \ \ \forall\ \ \ i\in V$
- $x_{ij} \leq q_{ij}\ \  \forall\ \ \ i,j\in A$
- $x_{ij} \geq 0 (\geq l_i)\ \ \ \forall\ \ \ i,j\in A$

formulaizone estesa:
in una topologia di rete cone questa:

img

ad ognmi arco indico c_ij e q_ij e per ogni nodo d_i

allora la fomulazione estesa è: (prendendo il parametro del costo)
$$\min z = 2x_{12} + 5x_{13} + 3x_{23} +3x_{24} + 3_{x32} + 4x_34} + 3x_{42}
s.t
per i = 1: $x_{12} + x_{13} = 10$
per i = 2: $x_23 + x_{24} - x_{12} - x_{42} - x_{32} = 0$
per i = 3: $x_{32} +x_{34} - x_{13} - x_{23} = -3$
per i = 4: $x_42} - x_{24} - x_{34} = -7$

guardando la capacita scriviamo anche:
$x_{12} \leq 8$
$x_{13} \leq 2$
$x_{23} \leq 4$
$x_{24} \leq 7$
$x_{32} \leq 4$
$x_{34} \leq 5$
$x_{42} \leq 7$

$x_{12}, $x_{13}, x_{23}, x_{24}, x_{32}, x_{34}, x_{42} \geq 0$



se un instanza del problema è ammissiibile, cioe se lo sono i dati del problema, allora:
- puo esistere una soluzione ottima
- puo esere unbounded se esiste qualeche arco (i, j) \in A con costo negatico c_{ij} < 0

img

poiche il costo totale negativo nel loop 2,4,5 allora abbiamo:
$$x_{12} = 1, x_{23} = 1, x_{24} = x_{45} = x_{52} = + \infty$$

siuponendo capacita infinita

importante condiione è quella di interezza: dice che se i dati d_i e le capacita q_{ij} sono interi, allora esiste una soluizone ottima non unbounded che è anche esse interacioe x_{ij \in N_0}, cioe i flussi ottimali sono interi.


abbiamo dei casi speciali del problema di flusso a socoto minimo:
- problem a di trasprto (rtansportation problem)
- prob di assegnamento (assignment problem)
- problema di cammino piu breve/rapido (shortesr/quickest path)
- problema di massimo flusso (aximum flow problem)




PROBLEMA DI TRASPORTO:
consideriamo la variante di simple commodity( supponiamo di avere un grafo diaprtito dove l'insieme dei vertici è composto da due insiemi disgiunti: sorgenti (fino a m) e pozzi (fino a n). ad ogni soegente è associata un s_1, .., s_m \geq 0 che definiscono la fornitura e ai pozzi invece abbiamo una qunatita b_1, ..., b_m che rappresenta la domanda. tra questi vertici nei due insiemi abbiamo na coppia di sorgente-pozzo e per ogni arco ij è associato un costo cij (costo i trasporto per unita di flusso)

img

allora il grafo G è dato da: G = (V=v_1 \cup V_2, A) con V_1 \cap V_2 = insieme vuoto e viene detto grafo rbipartito ed è questo problemaun caso speciale del problema di flusso a caso minimo, nel qule abbiamo che:
- non abbiamo vincoli di capacita
- non abbiamo nodi di transito

il problema sara allora ammissiible se:
$$\sum_{i\in V_1} s_i = \sum_{i\in V_2} b_i$$

allora le varaibili abbiamo che:
$$x_{ij}, i\inV_1, j\in V_2$$ che indica qunate unuta di flusso ("merce") vengono gtraspostate da i a j


l'obiettivo è minimizzare al f.o. quindi la misura di prestazione sara: $\min z = \sum_{(i,j)\in A}\sum_{j\inV_2} c_{ij}x_{ij}$ alora minimizziamko il costo totale.

vincoli:
$$\sum_{j\in V_2} x_{ij} = s_i\ \ \ \forall\ \ \ i\in V_1$$ per ongi vertice sorgente
$$\sum_{i\in V_1} x_{ij} = b_i\ \  \\forall\ \ \ j\in V_2$$ per ongi vertice pozzo
$$x_{ij} \geq 0\ \ \ \forall\ \ \ i\in V_1, j\in v_2$$



PROBLEMA DI ASSEGNAMENTO LINAERE:
abbiamo sempre un siseme di vertici diviso in due grupi: n risorse e n tasks
obiettvo assengare le risorse ai tasks

img

assegnamo iun costo sll'arco ij con c_{ij}.

ad una risorsa possiamo associare un solo task.
es: magazzini (impianti) automatizzati: n = 4 con n risorse = ABV (automated guided vehicle)
abbiamo dei task da compiere come lo spostamente di un veicolo da un punto ad un altro

img

avremo allora i costi di oni veicolo per spostarsi allinizio del task
costi = tempo di viaggio dal puno di prelievo al punto di inizio

quindi questa tipologi a di problemi è un acso speciale del prtiblema di trasposto dove cambia solo che:
- m= n
- s_i = 1 \forall i
- b_i = 1 \forall j


le variabili saranno allora:
x_{ij} = 1 se la rosorsa i è aasegnata ad un task j
= 0 altrimenti

vorremo allora minimizzare la f.o.: $\min z = \sum_{i=1}^n\sum_{j=1}^n c_{ij} x_{ij}$
s.t.
- $\sum_{j=1}^n x_{ij} = 1\ \ \ \forall\ \ \i=1,...,n$ cioe ogni risrosa deve essere assegna ad un solo task
- $\sun_{i=1}^n x_{ij} = 1\ \ \ \forall\ \ \ j=1,...,n$ cioe ongi task deve essere assegnato ad una sola risorsa

dato che s_i = 1 con i = 1,...,n e b_j = 1 con j=1,...,n

possiamo imporre che x_{ij} \geq 0 impedendo che la variabile sia <= 0 qusto grazie all;interezza dei dati e ai vincoli qui sopra che impediscono che le varaibili assumano valore > 1 quidni potremo avere con valore 0 o 1.










