\newpage
\section{Databases}

% Definizioni di base
\subsection{Definizioni di base}

Le definizioni di base da sapere sono:
\begin{itemize}
	\item \hl{dato}: \textbf{insieme di fatti conosciuti, registrati e con un significato}. È detto dato grezzo visto che si suppone che andrò ad elaborarlo, questo dato \textbf{sara' poi archiviato}, sarà un \textbf{fatto conosciuto} cioè avremo:
		\begin{itemize}
			\item \textbf{eventi con un significato} per un dato tipologia di utenti 
			\item sorgente che \textbf{produce i dati} con una cerca velocità
		\end{itemize}
	
	\item \hl{DataBase}: \textbf{raccolta di dati altamente organizzati, intercorrelati e strutturati}. È una struttura con dei collegamenti strutturati tra i dati

	\item \hl{DBMS Data Base Managment System}: insieme di \textbf{programmi per accedere ai dati e farci delle operazioni} di 4 tipi: creazione, recupero, aggiornamento e cancellazione, ciclo \textbf{CRUD}. Ne favorisce anche il mantenimento.

	\item \hl{mini-world}: \textbf{parte del mondo reale alla quale si riferiscono i dati presi} andando a \textbf{limitare la modellazione} in un numero n di concetti

	\item \hl{DataBase System}: insieme di DBMS con i dati
	
	\item \hl{astrazione}: \textbf{separare i dati dai collegamenti tra le entità per disporle in un modello} senza che esso si occupi di come salvare i dati
	
	\item \hl{modello concettuale}: formato da entità e relaizoni
	
	\item \hl{modello fisico}: definizione dei tipi dato e dove sono \textbf{conservati}
	
	\item \hl{controllo della concorrenza}: garantire che tutte le transazioni sono \textbf{correttamente eseguite}
	
	\item \hl{recovery}: se la transazione è stata eseguita è stata conservata nel database
\end{itemize}


% Tipologie di DB
\subsection{Tipologie di DB}

Esistono molti tipi di DB:

\begin{itemize}
	\item numerici o testuali
	\item multimediali
	\item Geographic Information Systems (GIS)
	\item Data Warehouses
\end{itemize}


% Ciclo di vita del DB
\subsection{Ciclo di vita del DB}

È opportuno vedere un \hl{concetto di base dei dati}, cioè il loro ciclo di vita. Il più semplice è:

\begin{enumerate}
	\item \textbf{acquisizione} (scattered data)
	\item \textbf{aggregazione} (integrated data)
	\item \textbf{analisi} (knowledge)
	\item finisce in un \textbf{applicazione} che genera dei "log data" che saranno poi acquisiti come scattered data
\end{enumerate}

Da un \hl{punto di vista computazionale} queste fasi si devono prendere in un altro modo:

\begin{enumerate}
	\item storage dei data
	\item formattazione e pulizia
	\item capire cosa dicono i dati
	\item[?)] se non mi bastano i dati che ho posso integrare dei dati
\end{enumerate}


% Livelli di un DB
\subsection{Livelli di un DB}

Quando si ha un DB abbiamo 3 livelli da considerare

\begin{enumerate}
	\item \hl{fisico}: dove sono \textbf{salvati i dati}
	\item \hl{logico}: indica come i dati sono \textbf{collegati tra loro}
	\item \hl{view}: \textbf{rappresentazione} che sarà diversa per ogni tipo di utente
\end{enumerate}


% DBMS Data Base Managment System
\subsection{Data Base Managment System}

Un DBMS offre l'opportunità di:

\begin{itemize}
	\item \textbf{salvataggio} dei dati
	\item \textbf{definizione} modelli dati
	\item \textbf{manipolazione} dei dati
	\item \textbf{processare} e condividere i dati
\end{itemize}

Per quanto riguarda l'\hl{interazione con i DB} avremo 2 strumenti:

\begin{itemize}
	\item \hl{query}: accede a parti differenti di dati e formula una richiesta
	\item \hl{transazioni}: legge dei dati ed aggiorna alcuni valori, salvandoli nel DB
\end{itemize}


% Mini-world
\subsection{Mini-world}

Avremo bisogno di \hl{identificare delle entita'}, cioè i \textbf{concetti di base} che rappresentano una parte delle cose che inseriremo nel DB relazionale. Poi andremo a \hl{connettere tra loro le entita'}, dette relazioni (\hl{relationships}) (ER), \hl{ne derivano delle tabelle dette relation}.

Il tutto \hl{da derivare dai requisiti} e non dall'esperienza personale.

Le tabelle create dalle entità conterranno i dati che ho a disposizione. Saranno divisi in:

\begin{itemize}
	\item righe (record)
	\item colonne (attributi)
	\item celle (dati grezzi)
\end{itemize}


Si verrà quindi a creare un \hl{catalogo} con vincoli, tipo di dati e la relazione di appartenenza degli attributi.
